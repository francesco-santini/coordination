\documentclass{llncs}
\usepackage{mathptmx}

\usepackage{amsmath,amssymb,amsxtra,amsfonts,cancel}
\usepackage{graphicx,paralist}
\usepackage{url}
\usepackage{tikz-cd}
\usetikzlibrary{trees, arrows}
\usepackage{xspace}
%\usepackage{hyperref}
\usepackage{setspace}
\usepackage{tikz}
%\usepackage{algorithm}
%\usepackage{algpseudocode}
\usepackage{textcomp}
\usepackage{soul}
\usepackage{listings}
\usepackage{mathtools}

\usepackage{todonotes}
% To disable notes without deleting them
%\usepackage[disable]{todonotes}

%\floatstyle{plain}
%\newfloat{myalgo}{tbhp}{mya}

\newenvironment{Algorithm}[2][tbh]%
{\begin{myalgo}[#1]
		\centering
		\begin{minipage}{#2}
			\begin{algorithm}[H]}%
			{\end{algorithm}
		\end{minipage}
	\end{myalgo}}
% to cut ------------------------------------------------------
%\usepackage{paralist}
%\usepackage[small]{caption}
%\usepackage{textcomp}
%\usepackage{times}
%\addtolength{\floatsep}{-5mm} \addtolength{\textfloatsep}{-5mm}
% -------------------------------------------------------------

\newtheorem{define}[theorem]{Definition}

\newtheorem{exa}[theorem]{Example}
\def\smallromani{\renewcommand{\theenumi}{\roman{enumi}}
        \renewcommand{\labelenumi}{(\theenumi)}}

%\def\bigodiv{{ \mathbf{\bigodot \hspace{-11pt} \boxempty \,\,}}}

\def\bigodiv{ {\text{ \large $\mathbf\odiv\hspace{-9.3pt} \div$}} }
\def\bigominus{ {\text{ \large $\mathbf\odiv\hspace{-9.3pt} -$}} }


%\defodiv{{ \odiv\hspace{-7.5pt} \div}}
\def\0{{\mathbf 0}}
\def\1{{\mathbf 1}}
\def\C{{\mathcal C}}
\newcommand{\rrarrow}{\longrightarrow}
\newcommand{\diag}[2]{d_{{#1}{#2}}}
\newcommand{\comment}[1]{}
\newcommand{\tell}{{\bf tell}}
\newcommand{\atell}{{\bf atell}}
\newcommand{\ask}{{\bf ask}}
\newcommand{\ostop}{{\bf stop}}
\newcommand{\retract}{{\bf retr}}
\newcommand{\rarrow}{\rightarrow}
\newcommand{\remove}{\rightarrow}
%introdotto per rimuovere le prove
\newcommand{\shortNoProof}[1]{ }

\def\ent{\vdash}
\def\monid{{\mathbf 0}}
\def\1{{\mathbf 1}}
\def\C{{\mathcal C}}
\def\K{{\mathcal K}}
\long\def\comment#1{}
\def\monop{\otimes}
\def\odiv{\, {\ominus\hspace{-7.7pt} \div} \,}
\def\monid{\mathbf{1}}

\newcommand{\SCCP}{\texttt{SCCP}\xspace}
\newcommand{\RefFig}[1]{Figure \nolinebreak\ref{#1}}
\newcommand\fnsep{\textsuperscript{,}}

%%%%%%%%%%%%%%%%%%%%%%%%%%%%%%%%%%%%%%%%%%%
%%%%%%%%%%%%%%%%%%%%%%%%%%%%%%%%%%%%%%%%%%%


\begin{document}

\title{Soft Concurrent Constraint Programming\\ with Local Variables
	%~\thanks{Research partially supported by the MIUR PRIN 2017FTXR7S ``IT-MaTTerS''.}
}


\author{Laura Bussi\inst{1}, Fabio Gadducci\inst{1}, 
Francesco Santini\inst{2}
} 
	\institute{Dipartimento di Informatica, University of Pisa, Italy \\
		\email{laura.bussi@phd.unipi.it} \qquad
		\email{fabio.gadducci@unipi.it}
		\and Dipartimento di Matematica e Informatica, University of Perugia, Italy\\
		\email{francesco.santini@unipg.it}
		}
	
\titlerunning{SCCP with Local Variables}
\authorrunning{Bussi, Gadducci, and Santini}

\maketitle

\begin{abstract}
???
\end{abstract}

\keywords{Soft concurrent constraint programming, local variables, observational semantics, polyadic algebras, residuated monoids.}

\section{Introduction}\label{sec:intro}
\emph{Concurrent Constraint Programming} (CCP) is a declarative model for concurrency where agents interact on a common store of information by telling and asking constraints~\cite{emerging}.  In its general meaning, a constraint is defined as a relationship on a set of variables: an assignment of (some of) the variables in the store needs to be found in order to satisfy a given goal.  A constraint system provides a signature from which the constraints can be constructed;  it can be formalised as an algebraic structure with operators to express conjunction of constraints, absence and inconsistent information, hiding of information and parameter passing.


The \emph{polyadic}  and  \emph{cylindric algebras} are two algebraisation of  the first-order calculus~\cite{cylalgebraic}, which have been widely  adopted in the literature to provide the semantics of constraint formulas~\cite{fgcs92,popl91}. A cylindric algebra is formed by enhancing a Boolean algebra by means of a family of unary operations called \emph{cylindrications}. Technically, the cylindrification operation $\exists_S(c)$ is used to project out the information about a set of variables $S$ from a constraint $c$: it is important to focus only on  the variables that appear in the goal of a constraint logic program, for example.

While polyadic algebras are the algebraic version of the pure first-order calculus, cylindric algebras yield an algebraisation of the first-order calculus with \emph{equality}. However, equality can be also achieved in polyadic algebras via additional axioms that specify which terms are to be considered equal under the abstract interpretation.

 While most of the solutions in the literature adopt a cylindric algebra to represent constraints~\cite{popl91}, other proposals take advantage of polyadic algebras: in \cite{fgcs92} the motivation is to allow projections on infinite sets, while in  \cite{festcatuscia} replacing diagonals (to perform parameters passing~\cite{popl91} and still borrowed from cylindric algebras) with polyadic operators allows for a compact – \emph{polynomial} – representation of soft constraints.  Moreover, in case it is necessary to use preferences beside hard constraints, i.e. \emph{Soft Concurrent Constraint Programming} (SCCP)~\cite{jacm97,jlamp17}, algebra operators interact with a residuated monoid structure of values~\cite{ipl17}: while the join semi-lattice of such preferences must be complete for cylindric algebras,  it is not necessary so for polyadic ones~\cite{festcatuscia}.

This work originates from \cite{festcatuscia}, where a first implementation of polyadic constraints was ideated (together with a polynomial representation of soft constraints). However, this paper extends that first attempt with a concurrent constraint language on top of its, and process-equivalence relations.
The work is organised as  follows: in Sect.~\ref{sec:bg} we present the necessary background on the algebraic structure needed to model polyadic constraints. Then, Sect~\ref{}, 

\section{An Introduction to Residuated Monoids}\label{sec:bg}

This section reports some results on residuated monoids,
which are the algebraic structure adopted for modelling
soft constraints in the following of the paper.
Results are mostly drawn from~\cite{jlamp17}, where also proofs can be found.


\subsection{Preliminaries on Ordered Monoids}\label{sec:lem}

The first step is to define an algebraic structure for modelling preferences,
where it is possible to compare values and combine them.
Our choice falls into the range of \emph{bipolar} approaches, in order 
to represent both positive and negative preferences: 
we refer to~\cite{ipl17} for a detailed introduction and 
a comparison with other proposals.

\begin{definition}[Orders]
	A partial order (PO) is a pair $\langle A, \leq \rangle$ such that
	$A$ is a set %of values 
	and $\leq \,\,\subseteq A \times A$ is a reflexive, transitive, and
	anti-symmetric  relation.
	% and $\forall a \in A. \bot\leq a$.
	%
	%A partial order with bottom (POT) is a triple
	%$\langle A, \leq, \bot \rangle$ such that $\langle A, \leq \rangle$ is a PO and
	%$\forall a \in A. \bot \leq a$.
	%
	A semi-lattice (SL) is a PO
	such that any non-empty finite subset of $A$ has a least upper bound (LUB).
\end{definition}

%We write 
The LUB of a (possibly infinite or empty) subset $X \subseteq A$ is denoted $\bigvee X$, and it is clearly unique.
Should  they exist, $\bigvee A$ and $\bigvee \emptyset$ correspond respectively to the top, denoted as 
$\top$, and to the bottom, denoted as $\bot$, of the PO.

\comment{\begin{definition}[Compact elements]
An element $a \in A$ is compact 
%(or finite) 
if whenever $a \leq \bigvee Y$ for some $Y \subseteq A$
there exists a finite subset
$X \subseteq Y$ such that $a \leq \bigvee X$.
%
%Let $A^C \subseteq A$ be the set of compact elements of ${\mathbb C}$.
%Then ${\mathbb C}$ is algebraic if $\forall c \in A. c = \bigvee \{ d \in A^C \mid d \leq c\}$.
\end{definition}

We let $A^C \subseteq A$ denote the set of compact elements of ${\mathbb C}$. }

%We considered the LUBs of possibly infinite sets just for the sake of simplicity: 
%our proposal would fit also the finite case.
%
%Obviously, Ls also have the greatest lower bound for any subset $Y \subseteq A$.
%In the following we fix a BL ${\mathbb L} = \langle A, \leq, \monid \rangle$.

%\begin{definition}[compact elements]
%An element $a \in A$ is compact (or finite) if whenever $a \leq \bigvee Y$ there exists a finite subset
%$X \subseteq Y$ such that $a \leq \bigvee X$.
%%
%%Let $A^C \subseteq A$ be the set of compact elements of ${\mathbb C}$.
%%Then ${\mathbb C}$ is algebraic if $\forall c \in A. c = \bigvee \{ d \in A^C \mid d \leq c\}$.
%\end{definition}


%Note that for complete lattices the definition of compactness given above coincides with the one using
%directed subsets. It will be easier to generalize it, though, to compactness with respect to the monoidal operator (see Def.~\ref{def:compactness}).
%
%We let $A^C \subseteq A$ denote the set of compact elements of ${\mathbb C}$. Note however
%that $A^C$ might be trivial: indeed, in the the segment $[0, 1]$ of the reals
%with the usual order, only $0$ is a compact element. As we are going to see, the situation for the soft paradigm
%can be more nuanced.
%\marginpar{is algebraicity needed?}
%

\begin{definition}[Ordered monoids]\label{defn:clm}
	A (commutative) monoid is a triple
	$\langle A, \monop,$ $\1 \rangle$ such that $\monop: A \times A \rightarrow A$ is
	a commutative and associative function and $\1 \in A$ is its \emph{identity} element,
	i.e., $\forall a \in A. a \monop \1 = a$.
%	
	A partially ordered monoid (POM) is a 4-tuple
	$\langle A, \leq, \monop, \1 \rangle$ such that 	
	$\langle A, \leq \rangle$ is a PO and $\langle A, \monop, \1 \rangle$ a monoid.
	%
	A semi-lattice monoid (SLM) is a 
	POM such that their underlying PO is a SL.
\end{definition}

As usual, we use the infix notation: $a \monop b$ stands for $\monop(a,b)$.
\comment{The monoidal operator can be defined for any multi-set: it is given 
for a family of elements $a_i \in A$ indexed over a non-empty finite
set $I$, and it is denoted by
$\bigotimes_{i \in I} a_i$.
%
If for an index set $I$ the $a_i$'s are different,
we write $\bigotimes S$ instead of $\bigotimes_{i \in I} a_i$
for the set $S = \{a_i \mid i \in I\}$.
%
Conventionally, we denote $\bigotimes \emptyset = \bot$.
}

\begin{example}[Power set]\label{ex:powerset}
	Given a (possibly infinite) set $V$ of variables, we consider
	the monoid $\langle 2^V, \cup, \emptyset \rangle$
	of (finite, possibly empty) subsets of $V$, with union as the monoidal operator.
	Since the operator is idempotent (i.e., $\forall a\in A.\, a \monop a = a$), 
	the natural order ($\forall a, b \in A.\, a \leq b$ iff $a \monop b = b$) 
	is a partial order, and 
	it coincides with subset inclusion:
	in fact, $\langle 2^V, \subseteq, \cup, \emptyset \rangle$
	is an SLM.
\end{example}

In general, the partial order $\leq$ and the multiplication operator $\otimes$ can be unrelated.
This is not the case for distributive SLMs.

\begin{definition}[distributivity]
\label{dist}
Let $\langle A, \leq, \monop, \monid \rangle$ be an SLM.
It is distributive if
	for  any  non-empty finite  $X \subseteq A$
%	\begin{itemize}
		it holds $\forall a \in A.\,  a \monop  \bigvee X = \bigvee \{a \monop x \mid x \in X\}$.
%	\end{itemize}
\end{definition}

Note that distributivity implies that $\otimes$ is monotone with respect to $\leq$.
\begin{remark}
% i.e., it holds
%	\begin{itemize}
%		%\item 
%		$\forall a, b, c \in A. a \leq b \implies c \monop a \leq c \monop b$.
%	\end{itemize}

	It is almost straightforward to show that our proposal encompasses many other formalisms in the literature.
	Indeed, distributive semi-lattice monoids are \emph{tropical} semirings (also known as dioids), 
	namely, semirings with an idempotent sum operator $a \oplus b$, which in our formalism is obtained as
	$\bigvee \{a, b\}$.
	% that is idempotent.
	%~\cite{tropical}. 
	If $\monid$ is the top of the SL we end up 
	in \emph{absorptive} semirings~\cite{golanShort}, 
	which are known as $c$-semirings 
	in the soft constraint jargon~\cite{jacm97} (see e.g.~\cite{ecai06} for a brief survey on residuation 
	for such semirings).
	Note that requiring the monotonicity of $\otimes$ and imposing $\monid$ to be the top of the partial order
	means that preferences are negative, i.e., 
	that it holds $\forall a, b \in A. a \monop b \leq a$.
\end{remark}

%\begin{example}
%Given a (possibly infinite) set $V$ of variables, two semi-lattice monoids are going to play a key role in the following sections. The first one is the semi-lattice monoid 
%$\mathbb{M}(V) = \langle 2^V_{fin}, \subseteq, \cup, \emptyset \rangle$
%of finite sub-sets of $V$, with the usual order given by sub-set inclusion.
%For the second one, we start by defining the support of an endofunction $f\colon V \to V$ as the set $sv(f) = \{ x \in V \mid f(x)\neq x \}$ and
%$F(V)$ as the set of functions $f\colon V \to V$ with finite support.
%The semi-lattice monoid of interest is  $\mathbb{F}(V) = \langle F(V), id, \circ, \iota \rangle$ where 
%$\iota$ is the identity function,  $\circ$ is function composition and $id$ is the discrete ordering on $F(V)$.
%\end{example}
%
%\bigskip
%
% COMMENTATO DA FILIPPO
%
%\begin{remark}
%The developments reported in Section~\ref{cypo} could be stated also for \emph{infinite} subsets 
%and for functions whose support is not necessarily finite. More on this later on.
%\end{remark}

%$a, b \in A$.
%
%The monoidal operator can be defined for any finite multiset: it is given for a family of elements
%$a_i \in A$ indexed over a finite set $I$, and it is denoted by
%$\bigotimes_{i \in I} a_i$.
%%
%Whenever for an index $I$ all the $a_i$'s are different,
%we simply write $\bigotimes S$ instead of $\bigotimes_{i \in I} a_i$
%for the set $S = \{a_i \mid i \in I\}$.
%%
%Conventionally, we will also usually denote $\bigotimes \emptyset = \top$.
%
%%smallskip
%%In the following we fix a IM ${\mathbb M} = \langle A, \monop, \monid \rangle$.
%
%We now move our attention to the domain of values we are going to consider.

\subsection{Remarks on Residuation}\label{sec:ror}
It is often needed to be able to ``remove'' part of a preference, due e.g. 
to the non-monotone nature of the language at hand
for manipulating constraints. 
%
The structure of our choice is given by residuated monoids~\cite{golanShort}. 
%
They introduce a new operator $\odiv$, which represents a ``weak'' (due to the presence of partial orders) inverse of $\otimes$.

\begin{definition}[residuation]\label{def:repo}
	A residuated POM is a 5-tuple $\langle A, \leq, \monop, \odiv, \monid \rangle$ such that
	$\langle A, \leq, \monop, \monid \rangle$ is a partially ordered monoid and
	$\odiv: A \times A \rightarrow A$ is a function satisfying $\forall a, b, c \in A. b \monop c \leq a \iff c \leq a \odiv b$.
%	An ReSL is an RePO such that the underlying PO is a SL.
\end{definition}

%In the following sections on oft CCP, we will often use absorptive RePOs, i.e., such that 
%	\begin{itemize}
%		\item[] $\forall a, \in A. a \leq 1$.
%	\end{itemize}
%
%However, 

%Residuation is monotone on the first argument: 
%$\forall a, b, c \in A. a \leq b \implies a \odiv c \leq b \odiv c$.
%Among other things, n
In order to confirm the intuition about weak inverses,
Lemma~\ref{rclm1} below precisely states that residuation conveys the meaning of 
an approximated form of subtraction.
% which can be used to remove a constraint from another.
% operator.
%
%We can give an order characterisation of the residuation operator.

\begin{lemma}\label{rclm1}
	Let $\langle A,$ $\leq, \otimes,  \odiv, \1 \rangle$ be a residuated POM with bottom.
	Then $a \odiv b = \bigvee \{c \mid b \otimes c \leq a\}$ for all $a, b \in A$.
\end{lemma}

In words, the LUB of the (possibly infinite) set 
$\{c \mid b \otimes c \leq a\}$ exists and is equal to $a \odiv b$.
%
In fact, residuation implies distributivity (see~\cite[Lemma 2.2]{ipl17}).

\begin{lemma}\label{rclm2}
	Let $\langle A, \leq, \monop, \odiv, \1 \rangle$ be a residuated POM. 
	Then $\monop$ is monotone.
	If additionally it is a SLM, then it is distributive.
\end{lemma}

\shortNoProof{
\begin{proof}
	By definition $a \odiv b$ is an upper bound of 
	$\{ c \mid b \monop c \leq a\}$ and $b \monop (a \odiv b) \leq a$.
	%
%	The latter property ensures the monotonicity of $\odiv$ on the first argument,
%	since by definition $a \odiv c \leq b \odiv c$ iff $c \monop (a \odiv c) \leq b$.
	%
%	As for the  monotonicity of $\monop$, it suffices to note that by definition
%	$a \leq (b \monop a) \odiv b$ and also by definition $c \monop a \leq c \monop b$ iff 
%	$a \leq (c \monop b) \odiv c$.
\qed
\end{proof}
}

%Note that by commutativity, 
%Thus $\monop$ is monotone (on both arguments), and
%the underlying monoid is ordered.
%while $\odiv$  is clearly anti-monotone on the second argument: 

%Distributivity holds also for the empty set and for infinite sets, if the LUBs exist.

[CAPIRE COSA SALVARE DA QUA ALLA FINE DELLA SEZIONE]

In order to ease the verification of the algebraic structure, it is often needed
a characterisation of residuation via simpler properties,
as the ones given below.

\begin{lemma}
\label{mono}
Let $\langle A, \leq, \monop, \monid \rangle$ be a POM  and
	$\odiv: A \times A \rightarrow A$ a function. Then $\langle A, \leq, \monop, \odiv, \monid \rangle$ is a residuated POM if and only if
	\begin{itemize}
		\item $\forall a, b \in A. b \monop (a \odiv b) \leq a \leq (b \monop a) \odiv b$,
		\item $\forall a, b, c \in A.\, a \leq b \implies a \otimes c \leq b \otimes c$ and $a\odiv c \leq b \odiv c$.
\end{itemize}
\end{lemma}

\shortNoProof{
\begin{proof} ($\Longrightarrow$)
The first item is immediate. Now, let $a \leq b$. Since $b \leq (b \otimes c) \odiv c$ and 
$c \otimes (a \odiv c) \leq a$, the second item follows.

($\Longleftarrow$)
Using the monotonicity of $\odiv$ from $b \monop c \leq a$ we get
 $(b \monop c) \odiv b \leq a \odiv b$, and by the first item
 $c \leq a \odiv b$.
 %
 From the latter by the monotonicity of $\otimes$ we get
 $b \otimes c \leq b \otimes (a \odiv b)$, and by the first item
 $b \monop c \leq a$.
 %
%	Immediate: $a \odiv b$ is an upper bound of 
%	$\{ c \mid b \monop c \leq a\}$ and $b \monop (a \odiv b) \leq a$.
	%
%	The latter property ensures the monotonicity of $\odiv$ on the first argument,
%	since by definition $a \odiv c \leq b \odiv c$ iff $c \monop (a \odiv c) \leq b$.
	%
%	As for the  monotonicity of $\monop$, it suffices to note that by definition
%	$a \leq (b \monop a) \odiv b$ and also by definition $c \monop a \leq c \monop b$ iff 
%	$a \leq (c \monop b) \odiv c$.
\qed
\end{proof}
}

It is easy to show that in any residuated POM the $\odiv$ operator is also anti-monotone on the second argument, i.e., 
$\forall a, b, c \in A.\, a\leq b \implies  c\odiv b \leq c \odiv a$.
%
Other properties are also straightforward, such as 
$\forall a\in A. \monid \leq a \odiv a$, which in turn implies 
that $\forall a\in A. a \monop (a \odiv a) = a$ and
%
%, and \emph{iii)} $a \odiv (b \monop c) = (a \odiv b) \odiv c$.
%should $\monop$ be idempotent, $b \leq a$ implies $a \odiv b = a$.
%
$\forall a, b \in A. a < b \implies \monid \not \leq a \odiv b$, where
$a < b$ means $a \leq b$ and $a \neq b$.
%
%Residuation is monotone on the first argument:
%$\forall a, b, c \in A. a \leq b \implies a \odiv c \leq b \odiv c$.
%%falsa and if $b \leq a$, then $a \odiv b = \monid$. For more properties of residuation we refer to \cite[Table~4.1]{resbook}.
%
%
The latter fact suggests the definition below, which identifies sub-classes 
of residuated monoids that are suitable for an easier manipulation
of constraints (see e.g.~\cite{ecai06}).

\begin{definition}[localisation / invertibility]
	A residuated POM $\langle A, \leq, \monop, \odiv, \monid \rangle$ is
	\begin{itemize}
		\item
		\emph{localised} if $\forall a, b \in A. a \leq b \implies a \odiv b \leq \monid$;
		\item
		\emph{invertible} if $\forall a, b \in A. a \leq b \implies b \monop (a \odiv b) = a$.
	\end{itemize}
\end{definition}

Note that if a residuated POM is localised then $\forall a \in A. a \odiv a = \monid$.
%\marginpar{all RePO are localized?}

%\begin{remark}
%	Note that the equivalence $a \otimes ((a \otimes b) \odiv a) = a \otimes b$ always holds, even if the 
%	underlying RePO is not invertible. Indeed, we have by definition $a \otimes ((a \otimes b) \odiv a) 
%	\leq a \otimes b$, as $(a \otimes b) \odiv a \leq b$. We must check that $a \otimes ((a \otimes b) 
%	\odiv a) \leq a \otimes b \iff b \leq ((a \otimes b) \odiv a) \iff a \otimes b \leq a \otimes b$,
%	which is trivially true.
%\end{remark}

\begin{remark}\label{rmk:soft}
	Some well-known structures used for soft constraints are the 
	%\emph{Boolean} ($\langle \{\mathit{false},\mathit{true}\}, \mathit{false} \leq \mathit{true}, \wedge, \mathit{false}, \mathit{true}\rangle$), 
	\emph{Fuzzy} ($\langle [0,1], \leq,$ $\min, 1 \rangle$), \emph{Probabilistic} ($\langle [0,1], \leq,\allowbreak\times, 1 \rangle$), 
	and \emph{Tropical}   ($\langle \mathbb{R}^+, \geq, +, 0 \rangle$) semirings, for $\geq$ the inverse of the standard order 
	(thus $0$ the top of the SL). In all these cases the underlying monoids 
	are both invertible and localised, thus
	%
	the $\odiv$ operator can be also used to
	(partially) relax constraints (see again~\cite{ecai06}).
\end{remark}

\shortNoProof{
\begin{proof} Let $X \subseteq A$ be a finite non-empty set. 	
	%\paragraph{$\bigvee \{a \monop x \mid x \in X\} \leq a \monop  \bigvee X$}
	\[\forall x \in X.\, x \leq \bigvee X %\implies %\forall x \in X.\, x \leq (a \monop \bigvee X) \odiv a \implies\]
	\implies \forall x \in X.\, a \monop x \leq a \monop \bigvee X \implies \bigvee \{a \monop x \mid x \in X\} \leq a \monop  \bigvee X .\]

	%So, let us assume that $X$ is inhabited.
	%\paragraph{$a \monop  \bigvee X \leq \bigvee \{a \monop x \mid x \in X\}$}
	\[\forall y \in X.\, a \monop y \leq \bigvee \{a \monop x \mid x \in X\} \implies 
	\forall y \in X.\, y \leq (\bigvee \{a \monop x \mid x \in X\}) \odiv a \implies\] 
	\[ \implies \bigvee X \leq (\bigvee \{a \monop x \mid x \in X\}) \odiv a \implies 
	a \monop \bigvee X \leq \bigvee \{a \monop x \mid x \in X\} .\] 
\qed
\end{proof}
}

\comment{
%Note that the proof does not require that $\otimes$ is monotone, which is thus a derived property.
%
Distributivity holds also for the empty set and for infinite sets, if the necessary LUBs exist.
%
Instead, it holds only partially for $\odiv$: this follows directly from the monotonicity of $\odiv$ on the first argument, 
since it implies that $x \odiv a \leq \bigvee X \odiv a$ for all $x \in X$.

\begin{lemma}
	\label{distodiv}
	Let $\langle A, \leq, \monop, \odiv, \monid \rangle$ be an ReSL and $X \subseteq A$ a finite non-empty set. Then 
	\begin{itemize}
		\item $\forall a \in A.\, \bigvee \{ x \odiv a \mid x \in X \} \leq \bigvee X \odiv a$
	\end{itemize}	
\end{lemma}

\shortNoProof{
\begin{proof}
Straightforward, since by the monotonicity of $\odiv$ in the first argument (Lemma~\ref{mono}) we get
% \[\forall x \in X.\,a \otimes (x \odiv a) \leq x \implies\]
 %\[\forall x \in X.\,a \otimes (x \odiv a) \leq \bigvee X \implies\]
 $\forall x \in X.\,x \odiv a \leq \bigvee X \odiv a$, which implies
 $\bigvee \{ x \odiv a \mid x \in X\} \leq \bigvee X \odiv a$.
% \[\]
\qed
\end{proof}
}

%\begin{remark}
Also this inequation holds for the empty set and for infinite sets, if the necessary LUBs exist.
%
Moreover, it also holds that $\bigvee \{ a \odiv x \mid x \in X \} \geq a \odiv \bigvee X$, since $\odiv$ is anti-monotone on the second argument.
%\end{remark}
}

\begin{proposition}\label{reabs}
	Let $\langle A, \leq, \monop, \odiv, \monid \rangle$ be a residuated SLM. The following are equivalent
	\begin{enumerate}
		\item $\forall a \in A.\, a \leq \1$
		\item $\forall a \in A.\, \1 \odiv a = \1$		
		\item $\forall a, b \in A.\, a \leq b \implies b \odiv a = \1$
	\end{enumerate}	
\end{proposition}

\shortNoProof
{
\begin{proof}
Note that $1$ immediately implies both $2$ and $3$, since by definition $a \leq b$ implies $\1 \leq b \odiv a$
and $\1$ is the top of the partial order.

For the second step, first note that both properties implies that $\1 \odiv a \leq \1$ for all $a \in A$. This is immediate
for $2$. As for $3$, %let us assume that $b \odiv a \leq \1$, and
consider $b = \bigvee \{\1, a \}$. By Lemma~\ref{distodiv} we have that
$\bigvee \{\1 \odiv a, a \odiv a\} \leq \bigvee \{\1, a \} \odiv a$. 
Hence, $\1 \odiv a \leq \1$ for all $a \in A$, and the result follows.

Finally, note that $\1 \odiv a \leq \1$ for all $a \in A$ implies that $\1 \odiv (\1 \odiv c) \leq \1$ for all $c \in A$, 
and since it always holds
that $c \leq \1 \odiv (\1 \odiv c)$, then $3$ implies $1$.
\qed
\end{proof}
}

%\begin{remark}
%In general, given an ReSL $\mathbb{M} = \langle A, \leq, \otimes, \odiv, \monid \rangle$, if $A \subseteq B$ such that $B(\otimes)$ is a group and $\odiv$
% is the inverse of $\otimes$, then $\mathbb{M}$ is fully $%\odiv$-distributive, which follows from $\mathbb{M}$ is distributive for $\otimes$. \\
%Consider, for istance, $\mathbb{M} = \langle \{0,...,5\},\geq,\oplus,\ominus,0 \rangle$, where $\oplus$ and $\ominus$ are the bounded sum and subtraction (e.g. $2 \oplus 4 = 5$, $2 \ominus 4 = 0$): 
%it is clear that, in this case, distributivity holds for $\ominus$, as long as $a \geq b \implies b \ominus a = 0$. \\
%\todo{un esempio dove $\odiv$ non distribuisce}
%In the following example it is shown that distributivity for $\odiv$ could hold partially, since we choose a residuation operator which is not the inverse of $\otimes$.
%\end{remark}

%\begin{remark}\label{rmk:softUnit}
%The proposition above provides an important characterisation for all absorptive ReSLs, including all those mentioned in Remark~\ref{rmk:soft}.
%\end{remark}

There are some important classes of residuated SLMs  such that $\odiv$ is easily proved to be distributive in the first argument,
while it is not so with respect to the second argument, not even in the absorptive case.

\begin{lemma}
	\label{distodiv2}
	Let $\langle A, \leq, \monop, \odiv, \monid \rangle$ be a residuated SLM such that $\langle A, \leq \rangle$ is a total order and $X \subseteq A$ a finite non-empty set. Then 
	\begin{itemize}
		\item $\forall a \in A.\, \bigvee \{ x \odiv a \mid x \in X \} = \bigvee X \odiv a$
	\end{itemize}	
\end{lemma}

\shortNoProof{
\begin{proof}
If $\langle A, \leq \rangle$ is a total order and $X$ is finite and non-empty we have that $\bigvee X \in X$, and since $\odiv$ 
is monotone on the first argument (see Lemma~\ref{mono}) the result follows.
\qed
\end{proof}
}

\begin{example}
\label{nodist2}
%Let $\langle A, \leq, \monop, \odiv, \monid \rangle$ be an ReSL such that $\langle A, \leq \rangle$ is a total order.
%Then, it holds that $\bigvee \{ x \odiv a \mid x \in X \} = \bigvee X \odiv a$ for all elements $a$ and (non-empty finite) subsets $X$.
%In fact, if $\langle A, \leq \rangle$ is a total order and $X$ is finite and non-empty we have that $\bigvee X \in X$, and since $\odiv$ 
%is monotone on the first argument (see Lemma~\ref{mono}), the result follows.
%
Let $n$ be a positive integer and $[n] = \{0, \ldots, n\}$ the segment of integers from $0$ to $n$. We can now define the (bounded) monoid $\mathbb{M}_n$ 
as the tuple $\langle [n], \geq, \oplus, \ominus, 0 \rangle$, where $\oplus$ and $\ominus$ are the bounded sum and subtraction, 
which are given as $m\oplus p = min\{n, m+p\}$ and $m\ominus p = max\{0,m-p\}$.

Now, it can be shown that $\mathbb{M}_n$ is an absorptive residuated SLM, and since it is a total order,
$\ominus$ is  distributive on the first argument.
%
However, it is not distributive on the second one. Consider an integer $m$ such that 
$m \neq n$ and the set $\{m, m+1\}$:
we then have that $(m+1) \ominus \bigvee\{m, m+1\} = 1$,
while instead $\bigvee\{(m+1) \ominus m, (m+1) \ominus (m+1)\} = 0$.
\end{example}

\comment{
\begin{example}
Given $A = \{0,a,b,c,d,e\}$, consider the following partial order:
	\begin{center}
		\begin{tikzpicture}
			\node (top) at (0,0)  {$0$};
			\node (a) [below of= top] {$a$};
			\node [below left of=a] (left) {$b$};
			\node [below right of=a] (right) {$c$};
			\node (d) [below right of=left] {$d$};
			\node (e) [below of=d] {$e$};
			\draw [thick] (top) -- (a);
			\draw [thick] (a) -- (left);
			\draw [thick] (a) -- (right);
			\draw [thick] (left) -- (d);
			\draw [thick] (right) -- (d);
			\draw [thick] (d) -- (e);
		\end{tikzpicture}
	\end{center}
and $\mathbb{M} = \langle A, \geq, \otimes, \odiv, 0 \rangle$, where $\otimes$ and $\odiv$ are defined as follows:
\begin{center}
	\begin{tabular}{@{} *{7}{c} @{}}
	\\ $\otimes$ \ & 0 \ & a \ & b \ & c \ & d \ & e
	\\ 0 \ & 0 \ & a \ & b \ & c \ & d \ & e
	\\ a \ & a \ & b \ & c \ & d \ & e \ & f
	\\ b \ & b \ & c \ & d \ & d \ & e \ & e
	\\ c \ & c \ & d \ & d \ & d \ & e \ & e
	\\ d \ & d \ & e \ & e \ & e \ & e \ & e
	\\ e \ & e \ & e \ & e \ & e \ & e \ & e
	\end{tabular}
\\	
	\begin{tabular}{@{} *{7}{c} @{}}
	\\ $\odiv$ \ & 0 \ & a \ & b \ & c \ & d \ & e
	\\ 0 \ & 0 \ & 0 \ & 0 \ & 0 \ & 0 \ & 0
	\\ a \ & a \ & 0 \ & 0 \ & 0 \ & 0 \ & 0
	\\ b \ & b \ & a \ & 0 \ & 0 \ & 0 \ & 0
	\\ c \ & c \ & a \ & 0 \ & 0 \ & 0 \ & 0
	\\ d \ & d \ & c \ & c \ & b \ & 0 \ & 0
	\\ e \ & e \ & d \ & b \ & c \ & a \ & 0
	\end{tabular}
\end{center}
Then $\mathbb{M}$ is an absorptive ReSL with $0$ the top of the partial order, since it behaves as the ReSL in the example above, except for $b$ and $c$: thus, in this case, $\bigvee \{b,c\} = a$. \\
It's now easy to show that $\odiv$ is not distributive for the first argument: $\bigvee{b \odiv a, c \odiv a} = a$ and $\bigvee\{b,c\} \odiv a = a \odiv a = 0$.
\end{example}

%
%We can proove $\bigvee X \odiv a = \bigvee \{ x \odiv a \mid x \in X \}$ under the following hypotesis.
%
%\begin{lemma}
%	\label{distodiv2}
%	Let $\langle A, \leq, \monop, \odiv, \monid \rangle$ be an ReSL.
%	If $a \otimes (x \odiv a) = x = (a \otimes x) \odiv a$, then $\bigvee X \odiv a = \bigvee \{ x \odiv a \mid x \in X \}$.
%\end{lemma}
%
%\begin{proof}
% \[\bigvee X = \bigvee \{ a \otimes (x \odiv a) \mid x \in X \} \implies\]
% \[\bigvee X = a \otimes \bigvee \{ x \odiv a \mid x \in X \} \implies\]
% \[\bigvee X \odiv a = (a \otimes \bigvee \{ x \odiv a \mid x \in X\}) \odiv a \implies\]
% \[\bigvee X \odiv a = \bigvee \{ x \odiv a \mid x \in X\}.\]
% \[\]
%\end{proof}
%
%[Qui avevi fatto una correzione, mettendo nell'ipotesi una disuguaglianza al posto della seconda uguaglianza, ma in quel modo non potrei provare l'ultimo passaggio, quindi te la rispedisco cos\`{i}. 
%Magari ne discutiamo quando ci vediamo.]
%\\

%Distributivity over $\bigvee$ implies that $\monop$ is
%monotone in both arguments.
%%as well as $\forall a \in A. a \monop \bot = \bot$.

%%
%In the following, we fix a BLIM ${\mathbb S} = \langle A, \leq, \monop \rangle$.
%%
%The next step is to provide a suitable notion of infinite composition. The definition below is taken from~\cite{CLIM}
%(but see also~\cite[p.~42]{golan}).
%
%\begin{definition}[infinite composition]
%Let $I$ be a (possibly countable) set of indexes. Then, the composition $\bigotimes_{i \in I} a_i$
%is defined as $\bigvee_{J \subseteq I} \bigotimes_{j \in J} a_j$ for all finite subsets $J$.
%\end{definition}

%%\marginpar{distributivity wrt. $\vee$ or wrt. $\wedge$ coincide?}
%Thanks to distributivity, we can show that
%$\bigotimes$ is monotone, i.e., $\forall j \in I. a_j \leq b_j \implies
%\bigotimes_{i \in I} a_i \leq \bigotimes_{i \in I} b_i$.

%We now extends the notion of compactness.
%
%\begin{definition}[$\monop$-compact elements]\label{def:compactness}
%An element $a \in A$ is $\monop$-compact (or $\monop$-finite) if whenever $a \leq \bigotimes_{i \in I} a_i$
%then there exists a finite subset $J \subseteq I$ such that $a \leq \bigotimes_{j \in J} a_j$.
%
%Let $A^\monop \subseteq A$ be the set of $\monop$-compact elements of ${\mathbb S}$. Then ${\mathbb S}$ is
%$\monop$-algebraic if $\forall c \in A. c = \bigotimes \{ d \in A^\monop \mid d \leq c\}$.
%\marginpar{now $\monop$-algebraicity is incorrect}
%\end{definition}

%We let $A^\monop \subseteq A$ denote the set of $\monop$-compact elements of ${\mathbb S}$.
%%
%It is easy to show that a compact element is also $\monop$-compact.
%%
%Indeed, the latter notion is definitively more flexible.
%%
%Let us consider again the segment $[0, 1]$ of the reals, yet now with the inverse of the usual order (as used
%in the probabilistic SCPs). Instead of the LUB, an alternative monoidal
%product can be just the multiplication.
%%
%Since any infinite multiplication tends to $0$, then all the elements are
%$\monop$-compact, except the top element itself, that is, precisely $0$.
%\marginpar{is $\monop$-algebraicity needed?}
}

\comment{
\subsection{On residuation and semirings}

We now consider \emph{semirings} equipped with a partial order~\cite[Chapter~2]{golanShort}.

\begin{definition}[semirings]
	A (commutative) semiring is a 5-tuple
	$\langle A, \monop, \monop, \monid, \1 \rangle$ such that $\langle A, \monop, \monid \rangle$
	and $\langle A, \monop, \1 \rangle$ are (commutative) monoids
	satisfying
	\begin{itemize}
		\item $\forall a \in A. a \monop \monid = \monid$
		\item $\forall a, b, c \in A. a \monop (b \monop c) = (a\monop b) \monop (a \monop c)$
	\end{itemize}
	An ordered semiring is a 6-tuple
	$\langle A, \leq, \monop, \monop, \monid, \1 \rangle$
        such that  $\langle A, \leq, \monop, \monid \rangle$ is an ordered monoid and 
   	$\langle A, \monop, \monop, \monid, \1 \rangle$ a semiring satisfying
	\begin{itemize}
			\item $\forall a, b, c \in A. a \leq b \wedge \monid\leq c \implies c \monop a \leq c \monop b$
	\end{itemize}
\end{definition}

We often use an infix notation, as $a \monop b$ for $\monop(a,b)$.


[A QUESTO PUNTO BISOGNA VEDERE QUALI DI QUESTE TRE PROPRIETA'
DELL'ordered SEMIRING POSSONO DISCENDERE DA QUELLE DELLA RESIDUAZIONE,
IN MODO DA AVERE GLI EQUIVALENTI DEI LEMMA 3 E 4]

[ MA 1 MI SERVE A QUALCOSA?]

\begin{definition}[residuation, II]
	A residuated semiring (ReS) is a 7-tuple $\langle A, \leq, \monop, \odiv, \monop, \monid, \1 \rangle$
	such that	$\langle A, \leq, \monop, \odiv, \monid \rangle$
	 is a residuated monoid and $\langle A, \leq, \monop, \monop, \monid, \1 \rangle$ an ordered semiring,
	  satisfying 
	\begin{itemize}
            ????
	\end{itemize}
	A residuated SSL (ReSSL) is an ReS such that the underlying PO is a SL.
\end{definition}

[COSA PUO' SERVIRE COME ASSIOMA?]

%In the following sections on oft CCP, we will often use absorptive RePOs, i.e., such that 
%	\begin{itemize}
%		\item[] $\forall a, \in A. a \leq 1$.
%	\end{itemize}
%
%However, 
}

\comment{
Indeed, there are many classes of absorptive and idempotent ReSLs such that $\odiv$ 
is not distributive in either arguments.

\begin{example}
\label{notdistr}
First of all, note that a complete sup-lattice $\langle A, \leq \langle$ (i.e., admitting a sup for all subsets 
of $A$) can be turned into a ReSL. Indeed, $\otimes$ is just the meet, so we have that 

\begin{itemize}
\item $a \otimes b = \bigvee \{c \mid c \leq a \wedge c \leq b\}$
\item $a \odiv b = \bigvee \{c \mid c \otimes b \leq a\}$
\end{itemize}


$$x \otimes y = \bigg \{\begin{array}{ll}
	\1 & \mbox{ if } y \leq x \\
	x & \mbox{ if } y = \1 \\
	\bot & \ otherwise
	\end{array}$$

both meetand divis




the bounded sum $\oplus$ is here idempotent, $0$ is still 
the identity. We can make it int
of three otherwise unrelated elements, 
so that for all elements $x$ we have $x \otimes x = \1 \otimes x = x$ \
and furthermore $a \otimes b = a \otimes c = b \otimes c  = \1$.

We now add the bottom element $\bot$, in order to obtain a complete lattice.
Then $\otimes$ is extended in the expected way, so that $\bot$ is absorbing.
%
The resulting semi-lattice monoid is absorptive and residuated, with $\odiv$ defined as

$$x \odiv y = \bigg \{\begin{array}{ll}
	\1 & \mbox{ if } y \leq x \\
	x & \mbox{ if } y = \1 \\
	\bot & \ otherwise
	\end{array}$$
%
Thus, $\odiv$ does not distribute, since 
$\bigvee \{a \odiv c, b \odiv c\}  = \bot < \1 = \1 \odiv c = \bigvee \{a, b\} \odiv c$.
\end{example}
}

\comment{
\begin{example}
\label{notdistr}
Let us consider the monoid $S = \langle \{p,u,n,t\}, \otimes_s, u \rangle$ (with $t$ the top 
of three otherwise unrelated elements): 
$p$ and $n$ intuitively represent the sign of an integer, $t$ tells us that 
the sign cannot be determined, $u$ is the zero
and $\otimes_s$ (which is idempotent) tells us the sign of the addition of two integers, so that 
for all elements $x$ we have
\[x \otimes_s x = u \otimes_s x = x \mbox{  and  } t \otimes_s x = p \otimes_s n = t\]
%
We now add the bottom, in order to obtain a complete lattice.
The $\otimes_s$ is extended in the expected way,  so that $\bot$ is absorbing.
%
Intuitively, $\bot$ states that an element is unsigned:
a pattern the reader familiar with abstract interpretation formalisms will recognise.

The resulting semi-lattice monoid is residuated, with $\odiv$ defined as

$$x \odiv y = \bigg \{\begin{array}{ll}
	t & y \leq x \\
	\bot & \ otherwise
	\end{array}$$
%
Thus, $\odiv$ does not distribute, since 
$\bigvee \{p \odiv n, u \odiv n\}  = \bot < \bigvee \{p, u\} \odiv n = t \odiv n = t$.
\end{example}
}

\section{An Alternative Proposal for Costraint Manipulation}
\label{newpro}

This section presents our personal take on polyadic algebras for ordered monoids:
the standard axiomatisation of e.g.~\cite{sagi2013} has been completely 
reworked, in order to be adapted to the constraints formalism.
%
We close the section by offering some preliminary insights on 
the laws for polyadic operators in residuated monoids.

\subsection{Cylindric and Polyadic Operators for Ordered Monoids}
\label{cypo}
We now introduce two families of operators 
%(cylindric and polyadic ones) 
that will be used
for modelling variables hiding and substitution, which represent
key features in languages for manipulating constraints.
%
One is a well-known abstraction for existential quantifiers,
the other an axiomatisation of the notion of
substitution, and it is proposed as a weaker  alternative 
to diagonals~\cite{popl91}, the standard tool for modelling 
equivalence in constraint programming.\footnote{``Weaker 
alternative'' here means that diagonals allow for axiomatising
substitutions at the expenses of working with complete
partial orders: see e.g.~\cite[Definition 11]{jlamp17}.}
%

\comment{\smallskip
Our first step is the introduction of a technical notion that allows for 
factorising the common properties in the definition of the two families of operators.

\begin{definition}[pomonoid action]
\label{pomo}
Let $\mathbb{M} = \langle A, \leq, \monop, \monid \rangle$ be a partially ordered monoid and $\mathbb{P} = \langle S, \leq \rangle$ a partial order.
A pomonoid action of $\mathbb{M}$ on $\mathbb{P}$ is a function $\phi: A \times S \rightarrow S$ such that
	\begin{itemize}
	     \item $\forall s \in S.\ \phi(\monid, s) = s$,
         \item $\forall a, b \in A,\ s \in S.\ \phi(a, \phi(b, s)) = \phi(a \otimes b, s)$,
         \item $\forall a, b \in A,\ s, t \in S.\ a \leq b\, \wedge\, s \leq t \implies \phi(a, s) 
         \leq \phi (b, t)$.
            % \item $\forall a, b \in A,\ s \in S.\ a \leq b \implies \phi(a, s) \leq \phi (b, s)$.
	\end{itemize}
\end{definition}

The first two requirements just state
that $\phi$ is a monoid action of $\mathbb{M}$ on $S$, while the latter states that $\phi$ is monotone. Sometimes, we say that $\mathbb{P}$ is an $\mathbb{M}$-PO.}

\subsubsection{Cylindric operators.}
We fix a POM $\mathbb{S} = \langle A, \leq, \monop, \monid \rangle$
and a set $V$ of variables, and we define a family of cylindric operators axiomatising existential quantifiers.

\begin{definition}[cylindrification]\label{cyli}
	A cylindric operator $\exists$ over $\mathbb{S}$ and $V$ ia family of monotone functions
	$\exists_x : A \rightarrow A$ indexed by elements in V such that for all 
	$a, b \in A$ and $x, y \in V$
	%\todo{$2_f^V$ non e' stato definito prima e/o f non si sa cosa e' qui}
	\begin{enumerate}
	     \item $a \leq \exists_x a$,
         \item $\exists_x \exists_y a = \exists_y \exists_x a$,
         %\item $\forall a, b \in A.\ X \subseteq Y\wedge a \leq b \implies  \exists(X, a) = \exists(Y, b)$,
	     %\item $\exists(X, \monid) = \monid$,
	     \item $\exists_x (a \monop \exists_x b) = \exists_x a \monop \exists_x b$.
	\end{enumerate}
	
	\noindent Let $a \in A$. The \emph{support} of $a$ is the set of variables 
	$sv(a) = \{ x \mid \exists_x a \neq a\}$. 
	% and the set of unsupported variables of $a$ is the set of variables $uv(a) =  V \setminus sv(a)$.
\end{definition}

In other words, $\exists$ fixes a monoid action which is monotone and increasing.

%Note that, since by Definition~\ref{pomo} we have $\exists(\emptyset, a) = a$, the requirements of Definition~\ref{cyli} trivially hold 
%whenever $X$ is the empty set.
%
%The first two conditions tell us that $\exists$ is a monoid action of $M(V)$ over $A$. Condition $3$ states
%that $\exists$ is a monotone function. Finally, the last two conditions state how $\exists$ interacts with the 
%monoidal structure on $\mathbb{S}$.
%
%\begin{remark}
%TODO bisogna vedere cosa altro serve, e se qualche propriet\`a \`e derivata.
%Cosa succede se $\mathbb{S}$ \`e un SL? Questo impatta sui LUB in M(V)?
%\end{remark}
%
%
%Note also that $\exists(X, \monid) = \monid$ would be a consequence of monotonicity,
%should $\monid$ be the top element. Also, the support is not necessarily finite.
%Finally, and importantly, note that 
%$X \cap sv(\exists(X, a)) = \emptyset$.

%\smallskip
%In the following, we often use $\exists_X a$ for $\exists(X, a)$, and $\exists_x a$ whenever $X = \{x\}$.

\subsubsection{Polyadic operators.}
We now move to define a family of operators axiomatising substitutions.  
They interact with quantifiers, thus, beside a partially ordered monoid $\mathbb{S}$
and a set $V$ of variables, we fix a cylindric operator $\exists$ over ${\mathbb S}$ and $V$.

As for notation, we let $F(V)$ be the set of functions with domain and codomain $V$.
For a function $\sigma: V \rightarrow V$ we let $supp(\sigma) = \{x \mid \sigma(x) \neq x\}$
and, given a set $X \subseteq V$, we denote by 
$\sigma \mid_{X}: X \rightarrow V$ the restriction and
by $\sigma^{c}(X)$ the counter-image of $X$ along $\sigma$.
%~\footnote{We are not going to need the other standard component proposed in the literature , i.e., \emph{diagonals}: a %family of elements $d_{x, y} \in A$ indexed by pairs of elements in $V$.}


\begin{definition}[polyadification]
	\label{def:poly}
	A polyadic operator $s$ for a cylindric operator $\exists$ is a a family of monotone functions 
	$s_\sigma: A \rightarrow A$
	indexed by elements in $F(V)$ such that for all $a, b \in A$, $x \in V$, and $\sigma, \tau\in F(V)$
	\begin{enumerate}
		\item $sv(\sigma) \cap sv(a) = \emptyset \implies s_\sigma a = a$
		\item $s_\sigma(a \monop b) = s_\sigma a \monop s_\sigma b$,
        \item $\sigma \mid_{sv(a)} = \tau \mid_{sv(a)} \implies s_\sigma a 
        = s_\tau a$,
        \item $\exists_x s_\sigma a = \begin{cases}
			s_\sigma \exists_y a &\text{if $\sigma^c(x) = \{y\}$}\\
			s_\sigma a &\text{if $\sigma^c(x) = \emptyset$}
			\end{cases}$.				
    \end{enumerate}
\end{definition}

%Clearly item $3$ always holds for an empty $X$.
%
A polyadic operator offers enough structure for modelling variable substitution. 
%
In the following, we fix a cylindric operator $\exists$
and a polyadic operator $s$ for it.

\comment{\begin{remark}
The laws are directly adapted from~\cite{sagi2013}, with the exception of $2$, which 
is stated as for a finite non-empty $X \subseteq V$ and $a \in A$
	\begin{itemize}
          \item[\emph{2'}.] $\sigma \mid_{V \setminus X} = \tau \mid_{V \setminus X}
		         \implies \forall a\in A.\ s(\sigma, \exists (X, a)) = s(\tau, \exists (X, a))$.
        \end{itemize}
However, the two formulations are equivalent. Indeed, note that
$\sigma \mid_{V \setminus X} = \tau \mid_{V \setminus X}$ implies 
$\sigma \mid_{sv(a) \setminus X} = \tau \mid_{sv(a) \setminus X}$, 
which in turn implies that 
$\sigma \mid_{\exists (X, a)} = \tau \mid_{\exists (X, a)}$, and 
assuming item $2$ the result follows.
%
For the vice-versa, first of all note that 
$\sigma \mid_{V \setminus X} = \tau \mid_{V \setminus X}$
coincides with $\sigma \mid_{Y \setminus X} = \tau \mid_{Y \setminus X}$
for $Y = sv(\sigma) \cup sv(\tau) \subseteq V$, and that $Y$ is finite
since both $\sigma$ and $\tau$ are finitely supported.
Now, $\sigma \mid_{sv(a)} = \tau \mid_{sv(a)}$ implies that 
$\sigma \mid_{Y \setminus (Y \setminus sv(a))} = \tau \mid_{Y \setminus (Y \setminus sv(a))}$,
thus by $2a$ we have 
$s(\sigma, \exists (Y \setminus sv(a), a)) = s(\tau, \exists (Y \setminus sv(a), a))$.
Since by definition we have $\exists (Y \setminus sv(a), a)) = a$, the result follows.
\end{remark}
}

%\begin{remark}
%Note also that $\sigma(\sigma^{c}(X)) \subseteq X$, so, when restricted to singleton, we have that item %$3$ in Definition~\ref{def:poly} is equivalent to
%\begin{itemize}
%          \item[\emph{3'}.] $\forall a\in A.\ \sigma^{c}(x) = \{y\} \implies \exists_x s_{\sigma} a =  %s_\sigma \exists_y a$,
%          \item[\emph{3''}.] $\forall a\in A.\ \sigma^{c}(x) = \emptyset \implies \exists_x s_{\sigma} %a =  s_\sigma a$.
%\end{itemize}
%\end{remark}

%\noindent As we did for $\exists$, we define the support of $\sigma$ as follows:
%\begin{itemize}
%\item $sv(\sigma) = \bigcap X \subseteq V \mid \sigma(X) \neq X$
%\end{itemize}

\subsection{Cylindric and Polyadic Operators for Residuated Monoids}
\label{cyre}
Both algebraic structures introduced in the previous section are quite standard,
even if polyadic operators are less-known in the soft-constraints literature:
we tailored their presentation to our needs, and indeed the properties
presented in Section~\ref{propo} appear to be original. It is now time to consider 
the interaction of such structures with residuation. 
%
To this end, in the following we assume that 
$\mathbb{S}$ is a RePO (see Definition~\ref{def:repo}).


\begin{lemma}
\label{divex}
Let $x \in V$ and $a, b \in A$.
%$X \subseteq V$ be finite. 
Then it holds
	%\begin{itemize}
         $\exists_x(a \odiv \exists_x b) \leq \exists_x a \odiv \exists_x b \leq
                                               \exists_x(\exists_x a \odiv b)$.
	%\end{itemize}
\end{lemma}

\begin{proof}
As for the inequality on the left, we have
 \[\exists_x b \otimes (a \odiv \exists_x b) \leq a \implies
   \exists_x(\exists_x b \otimes (a \odiv \exists_x b)) \leq \exists_x a \implies\]
 \[\exists_x b \otimes \exists_x(a \odiv \exists_x b)) \leq \exists_x a \implies
   \exists_x(a \odiv \exists_x b) \leq \exists_x a \odiv \exists_x b\]

For the inequality on the right, note that $\odiv$ is anti-monotone on the second argument and $\exists$ 
is increasing, so that $\exists_x a \odiv \exists_x b \leq \exists_x a \odiv b \leq \exists_x (\exists_x a \odiv b)$.
\qed
\end{proof}

\begin{remark}
\label{remdiv}
Looking at the proof above, it is clear that $\exists_x(a \odiv \exists_x b) \leq \exists_x a \odiv \exists_x b$
is actually equivalent to state that
$\exists_x(a \monop \exists_x b) \geq \exists_x a \monop \exists_x b$.
\end{remark}

We can show that $\odiv$ does not substantially alter the free variables of its arguments.

\begin{lemma}
Let $a, b \in A$. Then it holds $sv(a \odiv b) \subseteq sv(a) \cup sv(b)$. 
\end{lemma}
\begin{proof}
Let us assume that $x \not \in sv(a) \cup sv(b)$. Then
\[\exists_x(a \odiv b) =  \exists_x(a \odiv \exists_x b) \leq \exists_x a \odiv \exists_x b \leq
  \exists_x(\exists_x a \odiv b) =\exists_x(a \odiv b)\]
 and since $\exists_x(a \odiv b) =  \exists_x a \odiv \exists_x b = a \odiv b$,
 the result follows.
\end{proof}

%\begin{remark}
%\todo{un esempio dove $\odiv$ non distribuisce}
%\end{remark}

%Similarly, it is easy to show that it holds $\forall a, b \in A.\ \exists_x(\exists_x a \odiv b) \leq \exists_x a \odiv \exists_x b$. 
%

A result similar to Lemma~\ref{divex} relates residuation and polyadic operators.

%the following lemma holds.
%\todo{Mettere motivazione Lemma?}

\begin{lemma}
Let $a, b \in A$ and $\sigma \in F(V)$. Then it holds
%\begin{itemize}
$s_\sigma (a \odiv b) \leq s_\sigma a \odiv s_\sigma b$.
%\end{itemize}
Furthermore, if $\sigma$ is invertible, then the equality holds.
%\begin{itemize}
%\item $\forall a,b \in A.\ s_\sigma (a \odiv b) = s_\sigma a \odiv s_\sigma b$.
%\end{itemize}
\end{lemma}

\begin{proof}
As for the inequality, we have
\[ a \otimes (b \odiv a) \leq b \implies s_\sigma [a \otimes (b \odiv a)] \leq s_\sigma b \implies \]
\[ s_\sigma a \otimes s_\sigma(b \odiv a) \leq s_\sigma b \implies s_\sigma (b \odiv a) \leq s_\sigma b \odiv s_\sigma a \]

As for the equality, $\sigma$ invertible implies $s_{\sigma^{-1}} s_{\sigma} a = a 
= s_{\sigma} s_{\sigma^{-1}} a$. Then we have
\[ s_\sigma b \otimes (s_\sigma a \odiv s_\sigma b) \leq s_\sigma a  \implies
   s_{\sigma^{-1}} (s_\sigma b \otimes (s_\sigma a \odiv s_\sigma b))
\leq s_{\sigma^{-1}} s_\sigma a  \implies \]
\[ s_{\sigma^{-1}} s_\sigma b \otimes s_{\sigma^{-1}} (s_\sigma a \odiv s_\sigma b) \leq a \implies
   b \otimes s_{\sigma^{-1}} (s_\sigma a \odiv s_\sigma b) \leq a \implies \]
\[ s_{\sigma^{-1}} (s_\sigma a \odiv s_\sigma b) \leq a \odiv b \implies
   s_\sigma (s_{\sigma^{-1}} (s_\sigma a \odiv s_\sigma b)) \leq s_\sigma (a \odiv b) \implies \]
\[ s_\sigma a \odiv s_\sigma b \leq s_\sigma (a \odiv b) \]
\end{proof}

\subsection{Polyadic Soft Constraints}\label{sec:softconstraints}
\label{subsec:inst} 
We are now ready to advance our proposal of soft constraints: it follows yet generalises \cite{scc},
whose underlying algebraic structure is the one of absorptive semirings.

\begin{definition}[(soft) constraints]\label{def:softconstraints}
	Let $V$ be a set of variables, $D$ a finite domain of interpretation
	and ${\mathbb S} = \langle A, \leq, \monop, \odiv, \monid \rangle$ a residuated SLM.
	A \emph{(soft) constraint} $c: (V \rightarrow D) \rightarrow
	A$ is a function associating a value in $A$ with each assignment
	$\eta: V\rightarrow D$ of the variables.
\end{definition}

In this section and in the following one, we denote by $\mathcal{C}$ the set of constraints that can be
built starting from chosen $\mathbb S$, $V$, and $D$. The application of a
constraint function $c:(V \rightarrow D) \rightarrow A$ to a variable
assignment $\eta:V\rightarrow D$ is denoted $c\eta$.  

A constraint involves all the variables in $V$, yet it may depend on
the assignment of a finite subset of them, called its support. For
instance, a binary constraint $c$ with $supp(c)=\{x,y\}$ is a function
$c: (V\rightarrow D)\rightarrow A$ that depends only on the
assignment of variables $\{x,y\}\subseteq V$, meaning that two
assignments $\eta_1, \eta_2: V \rightarrow D$ differing only for the
image of variables $z \not \in \{x,y\}$ coincide (i.e., $c\eta_1 =
c\eta_2$).
%
%The support corresponds to the classical notion of scope of a
%constraint.  We often refer to a constraint with support $X$ as $c_X$.
%Moreover, an assignment over a support $X$ of cardinality $k$ is concisely
%represented by a tuple $t$ in $D^k$, and we often write $c_X(t)$
%instead of $c_X\eta$.

\smallskip
The set of constraints forms a residuated SLM, with the structure
lifted from ${\mathbb S}$.

\begin{lemma}[the ReSL of constraints]\label{prop:soft}
	The ReSL of constraints $\mathbb C$ is
	defined as the tuple $\langle {\mathcal C}, \leq, \monop, \odiv, \monid \rangle$ such that
	
	\begin{itemize}
		\item $c_1 \leq c_2$ if $c_1\eta\leq c_2\eta$ for all $\eta: V \rightarrow D$,
		\item $(c_1\monop c_2)\eta = c_1\eta\monop c_2\eta$, %for $c_1, c_2\ \in {\mathcal C}$
		\item $(c_1\odiv c_2)\eta = c_1\eta\odiv c_2\eta$, %for $c_1, c_2\ \in {\mathcal C}$
		\item $\monid \eta = \monid$.
	\end{itemize}
\end{lemma}


Combining constraints by the $\monop$ operator
means building a new constraint whose support involves at most
the variables of the original ones. The resulting constraint  associates with
each tuple of domain values for such variables the element
that is obtained by multiplying  those associated by the
original constraints to the appropriate sub-tuples.
%
%Residuation works as expected (i.e., $(c_1\odiv c_2)\eta = c_1\eta\odiv c_2\eta$),
%and 
%Also, the bottom is the constant function mapping all $\eta$ to $\bot$.

%\begin{example}[A simple CLIM]\label{execlim}
%Let us consider a CLIM $\mathbb S$, 
%and as $D$ a finite subset of the elements of the CLIM.
%A polynomial with variables in $V$ 
%and elements of the CLIM as coefficients
%such as $ux \, \hat{+} \, vy \, \hat{+} \, z$
%can be interpreted as the soft constraint associating 
%with a function $\eta: V \rightarrow D$ the value 
%$\bigvee \{u \monop \eta(x), v \monop \eta(y), z \}$.
%The composition of such constraints is straightforward, while 
%the ordering might not be the one induced by the coefficients, 
%due to the presence of constants.
%
%More precisely, let us consider the CLIM of non-negative reals and
%the polynomials $2x \, \hat{+} \, 1$ and $x \, \hat{+} \, 6$
%and let us assume $D = \{1, 2, 3\}$.
%The composition of such constraints is actually given just by coefficient 
%addition, so that
%$(2x \, \hat{+} \, 1) \monop (x \, \hat{+} \, 6) = 
%(3x \, \hat{+} \, 7)$.
%However, note that $2x \, \hat{+} \, 1 \leq x \, \hat{+} \, 6$.
%
%
%Similarly for residuation, which is just bounded subtraction of coefficients.
%Since $2x \, \hat{+} \, 1 \leq x \, \hat{+} \, 6$,
%by construction ($2x \, \hat{+} \, 1) \odiv (x \, \hat{+} \, 6)$ is the bottom constraint,
%mapping all variables to $0$.
%Instead, $(x \, \hat{+} \, 6) \odiv (2x \, \hat{+} 1)$ could be described as $\hat{-}x \, \hat{+} \, 5$,
%even if
%the latter falls outside of the polynomials we considered since it has a negative coefficient:
%it suffices to assume that if the actual result of the evaluation of the polynomial is negative 
%then it is put to $0$.
%
%%
%If $D$ is not the singleton, the support of a polynomial is precisely the set of variables occurring in it.
%\end{example}

%The ReSL of constraints also enjoys the cylindric properties, as shown by
%the result below (for cylindric operators and diagonals in the idempotent case, see~\cite{scc}).

\begin{lemma}[Cylindric and polyadic operators for (soft) constraints]
	The ReSL of constraints $\mathbb{C}$ admits cylindric and polyadic operators, defined as
	\begin{itemize}
		\item  $(\exists_x c) \eta = \bigvee \{c \rho \mid \eta\mid_{V \setminus \{x\}} = 
		\rho\mid_{V \setminus \{x\}}\}$ for all $c \in {\mathcal C}, x \in V$
		%\item if $\sigma$ is an injective substitution, then $(s_{\sigma}c)\eta = c(\sigma \circ \eta)$ 
		%for all $c \in \mathcal{C}$
		\item  $(s_\sigma c) \eta = c (\eta \circ \sigma)$ for all $c \in {\mathcal C}, \sigma \in F(V)$	
%		\item $\delta_{x,y}\eta = \left\{
%		\begin{array}{rcl} \bot & & \text{if } \eta(x) = \eta(y); \\
%		\top & & \text{otherwise.}
%		\end{array} \right.$ for all $x, y \in V$
	\end{itemize}
\end{lemma}

Hiding means eliminating variables from the support:
$supp(\exists_x c) \subseteq supp({c}) \setminus {x}$.\footnote{The operator
	is called \emph{projection} in the soft framework,
	and $\exists_x c$ is denoted $c\Downarrow_{V\setminus \{x\}}$.}

\begin{proof}[TO BE REDONE]
%Let us consider a finite set $X$ and a constraint $c$. Recall that since $supp(c)$ is finite,
%$c$ can be considered as a function $D^k \rightarrow A$ for $k = \# supp(c)$.
%
%Thus $(\exists_X c) \eta$ just boils down to consider the LUB of $\# supp(c) \setminus X$ 
%sets of $k$-tuples, and t
The properties of the pomonoid action (see Definition~\ref{pomo})
are easily shown to hold for both operators. 
%
As for the cylindric laws (see Definition~\ref{cyli}), first note that the set of functions 
$\rho$ such that $\eta\mid_{V \setminus \{x\}} = \rho\mid_{V \setminus \{x\}}$ is actually finite.
Thus, we have that 
\begin{eqnarray*}
(\exists_x (c \otimes \exists_x d)) \eta & = & \bigvee_\rho \{(c \otimes \exists_x d) \rho \mid \eta\mid_{V \setminus \{x\}} = \rho\mid_{V \setminus \{x\}}\}\\
& = & \bigvee_\rho \{c\rho \otimes (\exists_x d) \rho \mid \eta\mid_{V \setminus \{x\}} = \rho\mid_{V \setminus \{x\}}\} \\
& = & \bigvee_\rho \{c\rho \otimes  (\bigvee_\xi \{d \xi \mid \rho\mid_{V \setminus \{x\}} = \xi\mid_{V \setminus \{x\}}\}) \mid \eta\mid_{V \setminus \{x\}} = \rho\mid_{V \setminus \{x\}}\} \\
& = & \bigvee_\rho \{c\rho \mid \eta\mid_{V \setminus \{x\}} = \rho\mid_{V \setminus \{x\}}\} \otimes \bigvee_\xi \{d \xi \mid \eta\mid_{V \setminus \{x\}} = \xi\mid_{V \setminus \{x\}}\} \\
& = & (\exists_x c) \eta \otimes (\exists_x d) \eta
\end{eqnarray*}

% = 
% =\]
%= \]
% =\]
% =
%(\exists_X c) \eta \otimes (\exists_X d) \eta \]
%%\[ (\exists_X d) \rho = \bigvee \{d \xi \mid \rho\mid_{V \setminus X} = \xi\mid_{V \setminus X}\}\]

[da rivedere]
Let us now move to the polyadic laws (see Definition~\ref{def:poly}).We just consider the third item, 
and we assume that $\sigma \mid_{\sigma^c(X)}$ is injective, thus
\begin{eqnarray*}
(\exists_X s_\sigma c) \eta & = &\bigvee_\rho \{(s_\sigma c) \rho \mid \eta\mid_{V \setminus X} = \rho\mid_{V \setminus X}\} = \bigvee_\rho \{c (\rho \circ \sigma) \mid \eta\mid_{V \setminus X} =   \rho\mid_{V \setminus X}\} \\ 
& = & \bigvee_\xi \{c\xi \mid (\eta \circ \sigma)\mid_{V \setminus \sigma^{c}(X)}  =  \xi\mid_{V \setminus \sigma^{c}(X)}\} \\
& = & (\exists_{\sigma^c(X)} c) (\eta \circ \sigma) = (s_\sigma \exists_{\sigma^c(X)} c) \eta
\end{eqnarray*}

%  \bigvee \{c (\rho \circ \sigma) \mid (\eta\circ \sigma)\mid_{V \setminus \sigma^{c}(X)} = (\rho\circ \sigma)\mid_{V \setminus \sigma^{c}(X)}\}=^{*}\]
\noindent
where it always holds that $\eta\mid_{V \setminus X} = \rho\mid_{V \setminus X}$ implies $(\eta\circ \sigma)\mid_{V \setminus \sigma^{c}(X)} = (\rho\circ \sigma)\mid_{V \setminus \sigma^{c}(X)}$,
while since $\sigma \mid_{\sigma^c(X)}$ is injective we have that
a $\xi$ satisfying $(\eta \circ \sigma)\mid_{V \setminus \sigma^{c}(X)} = \xi\mid_{V \setminus \sigma^{c}(X)}$ can be decomposed as $\rho\circ \sigma$
for a $\rho$ such that $\eta\mid_{V \setminus X} = \rho\mid_{V \setminus X}$
(otherwise, it could happen that for some $\{x, y\} \subseteq \sigma^c(X)$ we have that $\sigma(x) =\sigma(y)$ and 
$\xi(x) \neq \xi(y)$).
\qed
\end{proof}

\section{Polyadic Soft CCP: syntax and reduction semantics}\label{sec:detSCCP}
This section introduces our (meta-)language.
We fix a set of variables $V$, ranged over by $x$, $y$, $\ldots$, and 
a residuated POM $\mathbb S = \langle {\mathcal C}, \leq, \otimes, \odiv, \1\rangle$, 
which is cylindric and polyadic over $V$ and whose elements
are ranged over by $c$, $d$, $\ldots$

\begin{definition}[Agents]%
The set $\mathcal{A}$ of agents, %which is
parametric with respect to a set $\mathcal{P}$ of (unary) procedure declarations $p(x) = A$,
is given by the following grammar
\[ A \Coloneqq \: \: \mathit{\ostop} \mid \textit{\tell}(c)  \mid \textit{\ask}(c) \mapsto A \mid A \parallel A \mid %\exists_x A \mid %Z \mid \mu_Z A 
p(x) \mid \exists_x A\]  
%for $\pi \in {\mathcal C}^\ast$ a (possibly empty) sequence of elements.
\end{definition}

In the following we will consider 
a set $\mathcal{E}$ of extended agents that uses the existential operator $\exists^{\pi}_x A$, 
where $\pi \in {\mathcal C}^\ast$ is meant to represent the sequence of updates performed on the local store. 
More precisely, the extended agent may carry some information about the hidden variable 
$x$ in an incremental way. We will often write $\exists_x A$ for $\exists^{[ ]}_x A$ and $\pi_i$ for 
the $i$-th element of $\pi = [ \pi_0, \ldots, \pi_n]$.

We denote by $fv(A)$ the set of free variables of an (extended) agent, defined in the expected way 
by structural induction, assuming that $fv(\tell(c)) = sv(c)$,
$fv(\ask(c) \mapsto A) = sv(c) \cup fv(A)$,
% and We also remark that $fv(\exists_x A) = fv(A) \setminus \{x\}$ 
and $fv(\exists^{\pi}_x A) = (fv(A) \cup \bigcup_i sv(\pi_i)) \setminus \{x\}$.
%
In the following, we restrict our attention to 
procedure declarations $p(x) = A$ such that $fv(A) = \{x\}$.


\begin{definition}[Substitutions]
Let $[^y/_x]: V \to V$ be the substitution defined as
\[ [^y/_x](w) = 
		\begin{cases} 
			y & \text{if $w = x$} \\
            w & \text{otherwise}
        \end{cases} \].

It induces an operator $[^y/_x]: \mathcal{E} \rarrow \mathcal{E}$ on extended agents as expected,  in particular

%\begin{itemize}
%	\item $\ostop[^y/_x] = \ostop$
%	\item $\tell(c)[^y/_x] = \tell(s_{[^y/_x]}c)$
%	\item $(\ask(c) \rightarrow A)[^y/_x] = \ask(s_{[^y/_x]}(c)) \rightarrow A[^y/_x]$
	%\item $[^y/_x] (\exists_w A)  = \exists_w ([^y/_x] A) \ \ \text{for $w \not \in \{x, y\}$}$
%	\item $p(w)[^y/_x] =  p([^y/_x](w))$
%	\item $(A_1 \parallel A_2)[^y/_x]  = (A_1[^y/_x] \parallel A_2[^y/_x])$
        $$(\exists^{\pi}_w A)[^y/_x] = \begin{cases} 
			\exists^{(s_{[^y/_x]} \pi)}_w A[^y/_x]  & \text{if } w \not \in \{x, y\} \\
             (\exists^{(s_{[^z/_w]} \pi)}_z A[^z/_w])[^y/_x] & \text{for } z \not \in fv(\exists^{\pi}_w A) \text{ otherwise}
        \end{cases}$$
%\end{itemize}
with $s_{[^y/_x]} [ \pi_1, \ldots, \pi_n ]$ a shorthand for $[s_{[^y/_x]} \pi_1, \ldots, s_{[^y/_x]} \pi_n]$.
\end{definition}

Note that the choice of $z$ in the rule above is immaterial, since for the polyadic operator it holds
$\exists_x c = \exists_y s_{[^y/_x]}(c)$ if $y \not \in sv(c)$.
%
In the following we consider terms to be equivalent up-to $\alpha$-conversion, meaning that terms 
differing only for hidden variables are considered equivalent, i.e.,
$\exists_w^\pi A = \exists_z^{(s_{[^z/_w]}\pi)} A[^z/_w]$ for $z \not \in fv(\exists^{\pi}_w A)$.

\begin{lemma}
Let $A \in \mathcal{E}$ and $x \not \in fv(A)$. Then $A[^y/_x] = A$.
\end{lemma}

%\begin{proof}
%For $\ostop$, $\tell(c)$, $p(w)$ the statement is trivially true, as we have that
%\begin{itemize}
%	\item $\ostop[^y/_x] = \ostop$
%	\item $\tell(c)[^y/_x] = \tell(s_{[^y/_x]}c) = \tell(c)$ by polyadic laws and $x \not
%	\in fv(\tell(c)) = sv(c)$.
%	\item $p(w)[^y/_x] = p(w)$ by $sv(p(w)) = {w} \wedge x \not \in sv(p(w)) \implies
%	w \not = x$.
%\end{itemize}
%As for the other agents, the statement can be proved by inductive reasoning, assuming it to be true for subagents in 
%$\ask(c) \rightarrow A$, $(A_1 || A_2)$ and $\exists^{\rho}_w A$. Indeed we obtain:
%\begin{itemize}
%	\item $(\ask(c) \rightarrow A)[^y/_x] =  
%	\ask(s_{[^y/_x]}c) \rightarrow A[^y/_x] = \ask(c) \rightarrow A$
%	by hypothesis and polyadic laws.
%	\item $(A_1 || A_2)[^y/_x] = (A_1[^y/_x] \parallel A_2[^y/_x])) = (A_1 \parallel A_2)$
%	by hypothesis.
%\end{itemize}
%As for $\exists^{\rho}_w A$, recall that $fv(\exists^{\rho}_w A) = (fv(A) \cup sv(\rho)) \setminus \{w\}$, thus 
%$x \not = w \wedge x \not \in fv(\exists^{\rho}_w A) \implies x \not \in fv(A) \wedge x \not \in sv(\rho)$. 
%Then we have two possible cases:
%\begin{itemize}
%	\item $w \not = x$. $(\exists^{\rho}_w A)[^y/_x] = \exists^{s_{[^y/_x]}\rho}_w A[^y/_x] = 
%	\exists^{\rho}_w A[^y/_x]$ (by $x \not = w$) $= \exists^{\rho}_w A$ (by hypotheses).
%	\item $w = x$. $(\exists^{\rho}_x A)[^y/_x] = ((\exists^{\rho}_z A)[^z/_x])[^y/_x] \ \ 
%	\text{for $z \not \in sv(\rho) \cup fv(A)$} = (\exists^{s_{[^y/_x]}(s_{[^z/_x]}\rho)}_z 
%	(A[^z/_x])[^y/_x]) = (\exists^(s_{[^z/_x]}\rho)_z A[^z/_x]) \text{by $x \not \in sv(c)
%	 \cup fv(A[^z/_x])$} \equiv_\alpha \exists^{\rho}_x A$.   
%\end{itemize}
%\end{proof}

\subsection{Reduction semantics}
We now move to the reduction semantics of our calculus. 
%
Given a sequence $\pi = [\pi_1, \ldots, \pi_n]$, we will use 
$\pi_\otimes$ and $\exists_x \pi$
as shorthands for $\pi_1 \otimes \ldots \otimes \pi_n$ and
$[\exists_x \pi_1, \ldots, \exists_x \pi_n]$, respectively,
sometimes combining them as in $(\exists_x \pi)_\otimes$,
with $[]_\otimes = \1$.

\begin{definition}[Reductions]\label{def:reductions}
Let $\Gamma = {\mathcal E} \times \C$ be the set of \emph{configurations}.
The \emph{direct reduction semantics} for SCCP is the pair 
$\langle \Gamma,  \to \rangle$
such that $\to \, \, \subseteq \, \,\Gamma \times   \Gamma$ is the 
binary relations obtained by the rules in 
Table~\ref{fig:operational}.

The \emph{reduction semantics} for SCCP is the pair 
$\langle \Gamma,  \rightarrow \rangle$
such that $\rightarrow \, \, \subseteq \, \,\Gamma \times   \Gamma$ is the
binary relation obtained by the rules in 
Table~\ref{fig:operational} and Table~\ref{fig:operational2}.
\end{definition}

%\vspace{-.25cm}
\def\odiv{\; {\ominus\hspace{-6pt} \div} \;}
\def\odivvv{\; {\ominus\hspace{-6pt} \div} \;}

\begin{table}  %\hfil5
  %\scalebox{0.9}{
   \begin{center}
   \begin{tabular}{lcll} 
   %
   \mbox{\bf A1}& $ {\displaystyle \langle \hbox{\tell}(c), \sigma \rangle \to \langle 
   \hbox{\ostop}, \sigma \otimes c\rangle}$
   \ \ \ & \bf{Tell}&
  \\ 
  &\mbox{   }&\mbox{   } &\mbox{   }
  \\
  \mbox{\bf A2}& $\frac {\displaystyle \sigma \leq c}{\displaystyle
  	\begin{array}{l} \langle \hbox{\ask}(c) \mapsto A, \sigma \rangle \to \langle A, \sigma \rangle   	\end{array}}$
    \ \ \ & \bf{Ask}&
    \\
    &\mbox{   }&\mbox{   }&
    \\
  \mbox{\bf A3}& $\frac {\displaystyle p(x) = A \in \mathcal{P} }
  {\displaystyle\langle p(y),\sigma\rangle \to \langle A[^y/_x], \sigma \rangle}$ 
  &\bf{Rec}&
    %\\
    %&\mbox{   }&\mbox{   }&
    %\\
    %\mbox{\bf A4}& $\frac {\displaystyle sv(\sigma) \cup fv(\exists_x A) \subseteq 
    %\Delta \wedge w \not \in \Delta }
    %{\displaystyle\langle \exists_x A,\sigma\rangle \to_\Delta \langle [^w/_x]A,
    %\sigma\rangle}$
    %&\bf{Hide}&
    %\\
   %&\mbox{   }&\mbox{   }&
  \end{tabular}
  \end{center}
\caption{Axioms of the reduction semantics for SCCP.}
\label{fig:operational}
\end{table}

\begin{table}  %\hfil5
  %\scalebox{0.9}{
   \begin{center}
   \begin{tabular}{lcll} 
   %
  \mbox{\bf R1}& $\frac {\displaystyle \langle A, \sigma\rangle \to \langle A', \sigma' \rangle} 
  {\displaystyle \begin{array}{l}
                          \langle A\parallel B, \sigma \rangle \to \langle A'\parallel B, \sigma' \rangle
                          \end{array}}$ 
    & \bf{Par1}&
  \\
  & \mbox{   }&\mbox{   }&
  \\
    \mbox{\bf R2}& $\frac {\displaystyle \langle A, \pi_\otimes \otimes \sigma_0 \rangle
    \to \langle B, \sigma_1 \rangle \text{ with } \sigma_0 = \sigma \odiv (\exists_x \pi)_\otimes}
    {\displaystyle\langle \exists^{\pi}_x A,\sigma\rangle \to \langle 
    \exists^{\pi \rho}_x B, \sigma \otimes \exists_x \rho
    \rangle \text{ with } \rho = \sigma_1 \odiv (\pi_\otimes \otimes \sigma _0)} \text{ for } x \not \in sv(\sigma)$
    &\bf{Hide}&
  \end{tabular}
  \end{center}
\caption{Contextual rules of the reduction semantics for SCCP.}
\label{fig:operational2}
\end{table}

The split distinguishes between the axioms and the rules guaranteeing the closure with respect to the parallel and existential operators. Indeed, rule {\bf  R1} models the interleaving of two agents in parallel, assuming for the sake of simplicity
that the parallel operator is associative and commutative, as well as satisfying $ \hbox{\ostop} \parallel A = A$.
%
%
In {\bf A1} a constraint $c$ is added to the store $\sigma$.
%, which in the next step will be $\sigma \otimes c$.
%
{\bf A2} checks if $c$ is entailed by  $\sigma$: if not, the computation is blocked.
%
Axiom {\bf A3} replaces a procedure identifier with the associated body, renaming the formal parameter with the actual one.
%$A[^y/_x]$ stands for the agent obtained by replacing all the occurrences of $x$ with $y$.
%
%Axiom {\bf A4} hides the variable $x$ occurring in $A$, replacing it  
%with a globally fresh variable,
%as ensured by $w \not \in \Delta$.
%The latter is more general than just requiring that 
%$w \not \in fv(\exists_x A) \cup sv(\sigma)$, since
%$\langle B, \rho \rangle   \rarrow_\Delta$ implies that 
%$fv(B) \cup sv(\rho) \subseteq \Delta$.\footnote{Our rule is  reminiscent of 
%$(8)$ in~\cite[p.~342]{popl91}.}

Let us instead discuss in some details the rule {\bf R2}.
The intuition is that if we reach an agent $\langle \exists^{\pi}_x A,\sigma\rangle$, then during the computation
a sequence $\pi$ of updates has been performed by the local agent and $(\exists_x \pi)_\otimes$ 
has been added to the global store. In order to evaluate $A$, the chosen 
store is 
$\pi_\otimes \otimes \sigma_0$ for $\sigma_0 = \sigma \odiv (\exists_x \pi)_\otimes$, thus removing 
the effect $(\exists_x \pi)_\otimes$ of the sequence of updates from the global store
and adding the cumulative effect $\pi_\otimes$ of the local store, which may carry information about $x$. 
Now, $\rho = \sigma_1 \odiv (\pi_\otimes \otimes \sigma _0)$ 
is precisely the information added by the step originating from $A$, which is then restricted and added to $\sigma$. 
On the local store we simply add that effect $\rho$ to the sequence of updates, with
$\pi \rho = [\pi_0, \ldots, \pi_n, \rho]$.

\begin{lemma}[On monotonicity]
\label{rmono}
Let $\langle A, \sigma \rangle \rightarrow \langle B, \rho \rangle$ be a reduction. 
Then $\rho = (\rho \odiv \sigma) \otimes \sigma$ and $fv(\langle B, \rho \rangle) \subseteq fv(\langle A, \sigma \rangle)$.
%\begin{enumerate}
%\item $\exists \sigma''.\  \sigma' = \sigma'' \otimes \sigma$
%\item $fv(\langle B, \sigma' \rangle) \subseteq fv(\langle A, \sigma \rangle)$
%\end{enumerate}
\end{lemma}
 
%By the properties of residuation, a witness of the equality in item $1$ is
%$\sigma' \odiv \sigma$.

\begin{proof}
Thanks to the equivalence $((a \otimes b) \odiv b) \otimes b = a \otimes b$, which holds in residuated POMs,
it is easy to check that the first law is preserved by reductions,
in particular by axiom \mbox{\bf A1} and rule \mbox{\bf R2}.
The second law is proved by induction on rule application.
\qed
\end{proof}

\begin{remark}
The choice for the resulting global store in rule {\bf R2}. could have been 
$\sigma_0 \otimes \exists_x(\sigma_1 \odiv \sigma _0)$, 
Indeed, the expression has a nicer appearance,
%Indeed, thanks to Lemma~\ref{rmono},
%we know that $\sigma_1 = \sigma_0 \otimes (\sigma_1 \odiv \sigma_0)$. Now,
%since it holds that $c \otimes ((c \otimes d) \odiv c) = c \otimes d$,
%we have 
%$$\sigma_0 \otimes \exists_x(\sigma_1 \odiv \sigma _0) = 
%\exists (\sigma_0 \otimes (\sigma_1 \odiv \sigma _0)) = 
%\sigma_0 \otimes \exists_x (\sigma_1 \odiv \sigma_0)$$.
%
and $\sigma_1 \odiv \sigma_0$ seems to better represent the cumulative effect of the
changes by the local agent.
However, this is a misleading impression, since the composition of the single updates
has to take into account the existential quantifications.
%In fact, the last local update corresponds in our choice to
%$\sigma_1 \odiv (\rho \otimes \sigma _0)$.

Let us consider constraints $c$ and $d$ such that $\exists_x c = \exists_x d = \monid$ and $c \otimes d = \bot$.
As far as our running example is concerned, it might be that $c = x \leq 1$ and $d = x \geq 5$. 
Now, let $A$ be the agent $\hbox{\tell}(c) \parallel \hbox{\tell}(d)$ and $\gamma$ the configuration 
$\langle \exists_x A, \monid \rangle$.
Assuming that $c$ is added to the store, with both choices we end up with the reduction
$\gamma \to \langle \exists_x^{[ c ]} \hbox{\tell}(d), \monid \rangle$.
With the alternative definition for the global store, we would then reach 
 $\langle \exists_x^{[c]} \hbox{\tell}(d), \monid \rangle \to \langle \exists_x^{[c,\bot \odiv c]} \hbox{\ostop}, \bot \rangle$,
 %since $\sigma_1 \odiv \sigma_0 = \bot \odiv \monid = \bot$,  
 thus lifting to the global store the unsatisfiability of the local one.
 Instead, with our choice we obtain 
 $\langle \exists_x^{[c]} \hbox{\tell}(d), \monid \rangle \to \langle \exists_x^{[c,\bot \odiv c]} \hbox{\ostop}, \exists_x (\bot \odiv c) \rangle$, 
% since $\sigma \otimes \exists_x(\sigma_1 \odiv (\rho \otimes \sigma _0)) 
%= \monid \otimes \exists_x ( \bot \odiv (c \otimes \monid))$.
Now, obviously $\bot \leq \bot \odiv c$: the equality may hold if e.g. the semiring does not have zero divisors, 
but this is not the case, since we assumed $c \otimes d = \bot$.
Going back to our running example, with $c = x \leq 1$ we have $\bot \odiv c = \emptyset \odiv x \leq 1 = x > 1$
and thus $\exists_x (\bot \odiv c) = \monid$.

Note also how the sequence in the local store is $[c,\bot \odiv c]$: $\bot \odiv c$ does not necessarily coincide with $d$.
However, it is the minimal information collapsing the local store, and since the semiring is not 
necessarily invertible, it is often the best result we can get.
\end{remark}

Let $\gamma = \langle A, \sigma \rangle$ be a configuration.
%
We denote by $fv(\gamma)$ the set $fv(A) \cup sv(\sigma)$ and by
$\gamma[^z/_w]$ the component-wise application of the substitution $[^z/_w]$.

\begin{definition}
A configuration $\langle A, \sigma \rangle$ is initial if $A\in \mathcal{A}$
and it is reachable if it can be reached by an initial configuration via a sequence of 
reductions.
\end{definition}

\begin{lemma}[On monotonicity, II]
\label{mono2}
Let $\langle A \parallel \exists_x^\pi B, \sigma \rangle$ 
be a reachable configuration.
Then $\sigma = (\sigma \odiv (\exists_x \pi)_\otimes) \otimes (\exists_x \pi)_\otimes$.
%\begin{enumerate}
%\item $\exists \sigma'.\  \sigma = \sigma' \otimes (\exists_x \pi)_\otimes$
%\end{enumerate}
\end{lemma}

\begin{proof}
We prove an equivalent property, namely, that there is $\sigma'$ such that 
$\sigma = \sigma' \otimes (\exists_x \pi)_\otimes$. 
%
It holds for initial configurations, and since it is clearly preserved by reductions, 
in particular by those obtained by axiom \mbox{\bf A1} and rule \mbox{\bf R2},
then we are done.
\qed
%
%As for \mbox{\bf A1}, it suffices to prove that $(\sigma \odiv a) \otimes a = \sigma$ implies  
%$((\sigma \otimes c) \odiv a) \otimes a = \sigma \otimes c$. Now, 
%$((\sigma \otimes c) \odiv a) \otimes a \leq \sigma \otimes c$ always holds. 
%For the reverse, by hypothesis and monotonicity  
%$(\sigma \odiv a) \otimes a \otimes c \geq \sigma \otimes c$, so it suffices to prove that 
%$((\sigma \otimes c) \odiv a) \otimes a \geq (\sigma \odiv a) \otimes a \otimes c$.
%This is in turn implied by $((\sigma \otimes c) \odiv a) \geq (\sigma \odiv a) \otimes c$.
\end{proof}
%As before, a witness of the equality in item $1$ is
%$\sigma \odiv (\exists_x \pi)_\otimes$.


\subsection{Saturated Bisimulation}\label{sec:saturated}
As proposed in \cite{pippo} for crisp languages, we define a barbed equivalence between two agents~\cite{barbed}.  
%
Intuitively, barbs are basic observations (predicates) on the states of a system, and in our case they correspond 
to the constraints in $\mathcal{C}$.

\begin{definition} [Barbs]
Let $\langle A, \sigma \rangle$ be a configuration and $c \in \mathcal{C}$
and we say that $\langle A, \sigma \rangle$ verifies $c$, or that $\langle A, \sigma \rangle \downarrow_c$ holds, if  $\sigma \leq c$.
\end{definition}

In other terms, satisfying a barb $c$ means that $\hbox{\ask}(c)$ must be enabled in $\langle A, \sigma \rangle$.
%
We now move to consider equivalence and, %Since \emph{barbed bisimilarity} is an equivalence already for CCP, 
along~\cite{pippo}
we propose the use of \emph{saturated bisimilarity}
%~\cite{barbedMontanari} has been proposed 
in order to obtain a congruence.
%
%Definition~\ref{def:strongsb} and Definition~\ref{def:weaksb} respectively provide the strong and weak definition of saturated bisimilarity.
%We say that $\gamma = \langle P, \sigma\rangle$ satisfies the barb $c$, written $\gamma \downarrow_c$,
%iff $\gamma \longrightarrow \gamma'$ and $\gamma' \downarrow_c$.
%\marginpar{Are barbs compact?}

\begin{definition}[Saturated bisimilarity]\label{def:strongsb} A saturated bisimulation is a symmetric relation $R$ on configurations such that whenever
%$(\gamma_1,\gamma_2) \in R$ with $\gamma_1 = \langle A, \sigma \rangle$
%and $\gamma_2 = \langle B, \rho \rangle$
$( \langle A, \sigma \rangle,\langle B, \rho \rangle) \in R$
\begin{enumerate}
\item if $\langle A, \sigma \rangle \downarrow_c$ then $\langle B, \rho \rangle \downarrow_c$;
\item if $\langle A, \sigma \rangle \to \gamma_1$ then there is $\gamma_2$ such that $\langle B, \rho \rangle \to \gamma_2$ and $(\gamma_1, \gamma_2) \in R$;
\item $(\langle A,\sigma \otimes d\rangle, \langle B,\rho \otimes d \rangle) \in R$ for  all $d \in \mathcal{C}$.
\end{enumerate}
We say that $\gamma_1$ and $\gamma_2$ are  saturated bisimilar ($\gamma_1  \sim_{\mathit{s}} \gamma_2$) if there exists a  saturated  bisimulation $R$ such that $(\gamma_1 , \gamma_2 ) \in R$. We write $A \sim_{\mathit{s}} B$ if $\langle A, \monid \rangle \sim_{\mathit{s}} \langle B, \monid \rangle$.
\end{definition}

Note that $\langle A, \sigma \rangle \sim_{\mathit{s}} \langle B, \rho \rangle$ implies
that $\sigma = \rho$.
\comment{
This fact does not hold if we move to weak relations. To this end, 
we let $\rightarrow$ denote the reflexive and transitive closure of $\to$.

\begin{definition} [Weak barbs]
Let $\langle A, \sigma \rangle$ be a configuration and $c \in \mathcal{C}$.
We say that $\langle A, \sigma \rangle$ weakly verifies $c$, or that $\langle A, \sigma \rangle \Downarrow_c$ holds, 
if  there exists $\gamma' = \langle B, \rho \rangle$ such that 
$\gamma \Rightarrow \gamma'$ and $\rho \leq c$.
\end{definition}

\begin{definition}[Weak saturated bisimilarity]\label{def:weaksb} A weak saturated bisimulation is a symmetric relation $R$ on configurations such that whenever
%$(\gamma_1,\gamma_2) \in R$ with $\gamma_1 = \langle A, \sigma \rangle$
%and $\gamma_2 = \langle B, \rho \rangle$
$( \langle A, \sigma \rangle,\langle B, \rho \rangle) \in R$
\begin{enumerate}
\item if $\langle A, \sigma \rangle \downarrow_c$ then $\langle B, \rho \rangle \Downarrow_c$;
\item if $\langle A, \sigma \rangle \to \gamma_1$ then there is $\gamma_2$ such that $\langle B, \rho \rangle \Rightarrow \gamma_2$ and $(\gamma_1, \gamma_2) \in R$;
\item $(\langle A,\sigma \otimes d\rangle, \langle B,\rho \otimes d \rangle) \in R$ for all $d \in \mathcal{C}$.
\end{enumerate}
We say that $\gamma_1$ and $\gamma_2$ are  weakly saturated bisimilar ($\gamma_1  \approx_{\mathit{s}} \gamma_2$) if there exists a  
weak saturated  bisimulation $R$ such that $(\gamma_1 , \gamma_2 ) \in R$. 
We write $A \approx_{\mathit{s}} B$ if $\langle A, \monid \rangle \approx_{\mathit{s}} \langle B, \monid \rangle$.
\end{definition}

The asymmetry is functional to later sections. However, it is clearly equivalent to the standard symmetric version.

\begin{lemma}[Weak saturated bisimilarity, 2]\label{def:weaksb2}
Weak saturated bisimilarity coincides with the relation 
obtained from Definition~\ref{def:strongsb} by replacing $\to$ with $\Rightarrow$ and $\downarrow_c$ with $\Downarrow_c$.
\end{lemma}


Since $\sim_{\mathit{s}}$ and $\approx_{\mathit{s}}$ are saturated bisimulations, they are clearly upward closed 
and they are also congruences. Indeed, a context $C[\cdot]$, i.e., an (extended) sagent with an placeholder $\cdot$,
 can modify the behaviour of a configuration only by adding constraints to its store. 
}
%Since $\sim_{\mathit{s}}$ is a saturated bisimulation, it is clearly closed with respect
%to the addition of constraints to a store. 
Moreover, it is also a congruence. Indeed, a context $C[\cdot]$, i.e., an (extended) agent with an placeholder $\cdot$,
can modify the behaviour of a configuration only by adding constraints to its store. 

\begin{proposition}
Let $A$, $B$ be agents and $C[\cdot]$ a context.
If $A \sim_{\mathit{s}} B$
then $C[A] \sim_{\mathit{s}} C[B]$.
%%and $\langle C[A],\sigma \rangle  \sim_{\mathit{s}} \langle C[B],\rho \rangle$.
%Let $\langle A,\sigma \rangle, \langle B,\rho \rangle$ be configurations
%and $C[\cdot]$ a context.
%If $\langle A,\sigma \rangle \sim_{\mathit{s}} \langle B,\rho \rangle$
%then $\langle C[A],\sigma \rangle \sim_{\mathit{s}} \langle C[B],\rho \rangle$.
%%and $\langle C[A],\sigma \rangle  \sim_{\mathit{s}} \langle C[B],\rho \rangle$.
%%
%%Moreover, the same property holds for $\approx_{\mathit{s}}$.
\end{proposition}

%Note that we are considering here also extended agents.
%and that for the result it is pivotal to require 
%the closure with respect to the addition of constraints.

%\begin{proof}
%By definition $\sim_{\mathit{s}}$ is upward closed, so we just need to check congruence.
%We then proceed by induction on rule application.
%[DA FINIRE. MA SERVIREBBE UNA CHIUSURA RISPETTO ALLE SOSTITUZIONI INVERTIBILI...]
%\qed
%\end{proof}
%
%\marginpar{to be proved for $\exists_x-$ (not needed for Prop.3)}


\section{Labelled reduction semantics}
The definition of $\sim_{\mathit{s}}$ 
%and $\approx_{\mathit{s}}$ are fully abstract, they are somewhat 
is unsatisfactory
because of the store closure, i.e., the quantification in condition \emph{3} of 
Definiton~\ref{def:strongsb}.
% and Definition~\ref{def:weaksb}.
This section presents a labelled version of the reduction semantics that will be used to 
define a suitable bisimulation that avoids such drawback.

\begin{definition}[Labelled reductions]
	Let $\Gamma = {\mathcal A} \times \C$ be the set of \emph{configurations}.
	The  \emph{labelled direct reduction semantics} for SCCP is the pair 
	$\langle \Gamma,   \xrightarrow{ }  \rangle$
	such that $\to \, \, \subseteq \, \,\Gamma \times \mathcal{C} \times \Gamma$ is the ternary
	relation obtained by the rules in Table~\ref{fig:ALTS}.
	
	The \emph{labelled reduction semantics} for SCCP is the pair 
	$\langle \Gamma,  \rightarrow \rangle$
	such that $\rarrow \, \, \subseteq \, \,\Gamma \times \mathcal{C} \times  \Gamma$ is the ternary relation
         obtained by the rules in Table~\ref{fig:ALTS} and Table~\ref{fig:CRLTS}.
\end{definition}

\def\odiv{\; {\ominus\hspace{-6.8pt} \div} \;}
\def\odivv{\; {\ominus\hspace{-5.2pt} \div} \;}
\def\odivvv{\; {\ominus\hspace{-7.8pt} \div} \;}
\begin{table}  %\hfil5
   \begin{center}
   	  \scalebox{0.9}{
   \begin{tabular}{lcll} 
   %
   \mbox{\bf LA1}& ${\displaystyle \langle \hbox{\tell}(c), \sigma \rangle \xrightarrow{\monid} 
   \langle \hbox{\ostop}, \sigma \otimes c\rangle}$
   \ \ \ & \bf{Tell}&
  \\ 
  &\mbox{   }&\mbox{   } &\mbox{   }
  \\
  \mbox{\bf LA2}& $\frac{\displaystyle \alpha \leq  c  \odiv  \sigma} {{\displaystyle
  	\begin{array}{l} \langle \hbox{\ask}(c) \mapsto A, \sigma \rangle \xrightarrow{\alpha}
  	\langle A, \alpha \otimes \sigma \rangle
  	\end{array}}}$
  \ \ \ & \bf{Ask}&
  \\
  &\mbox{   }&\mbox{   }&
  \\
  \mbox{\bf LA3}& $\frac {\displaystyle p(x) = A \in  \mathcal{P} }
  {\displaystyle\langle p(y),\sigma\rangle \xrightarrow{\monid} \langle  A[^y/_x], \sigma\rangle}$ 
  &\bf{Rec}&
 \end{tabular}
}
  \end{center}
\caption{Axioms of the labelled semantics for \SCCP.}
\label{fig:ALTS}
\end{table}
\def\odiv{\, {\ominus\hspace{-7.8pt} \div} \,}
\def\odivvv{\; {\ominus\hspace{-4.7pt} \div} \;}


\def\odiv{\; {\ominus\hspace{-6.8pt} \div} \;}
\def\odivv{\; {\ominus\hspace{-5.2pt} \div} \;}
\def\odivvv{\; {\ominus\hspace{-7.8pt} \div} \;}
\begin{table}  %\hfil5
   \begin{center}
   	  \scalebox{0.9}{
   \begin{tabular}{lcll} 
   %
  \mbox{\bf LR1}& $\frac {\displaystyle \langle A,\sigma \rangle \xrightarrow{\alpha} \langle A', \sigma' \rangle} 
  {\displaystyle \begin{array}{l}
                          \langle A\parallel B, \sigma \rangle \xrightarrow{\alpha} \langle A'\parallel B, \sigma' \rangle
                          \end{array}}$ 
    & \bf{Par}&
  \\
  & \mbox{   }&\mbox{   }& \mbox{   }
  \\
  \mbox{\bf LR2} & $\frac {\displaystyle \langle A, \pi_\otimes \otimes \sigma_0 \rangle \xrightarrow{\alpha}
  \langle B, \sigma_1 \rangle \text{ with } \sigma_0 = \sigma \odiv (\exists_x \pi)_\otimes }
  {\displaystyle \langle \exists^\pi_x A, \sigma \rangle \xrightarrow{\alpha}
  \langle \exists^{\pi \rho}_x B, \alpha \otimes \sigma \otimes \exists_x \rho \rangle  \text{ with } \rho = \sigma_1 \odiv (\alpha \otimes \pi_\otimes  \otimes \sigma_0)}
    \text{ for } x \not \in sv(\sigma) \cup sv(\alpha)$
&\bf{Hide}&
  \end{tabular}
}
  \end{center}
\caption{Contextual rules of the labelled semantics for \SCCP.}
\label{fig:CRLTS}
\end{table}

%The split distinguishes between the axioms and the rules guaranteeing the closure with respect to the parallel and 
%existential operators. Indeed, rules {\bf  R1} models the interleaving of two agents in parallel, assuming for the sake of 
%simplicity that the parallel operator is associative and commutative, as well as $ \hbox{\ostop} \parallel A = A$.
%%
%%
%In {\bf A1} a constraint $c$ is added to the store $\sigma$.
%%, which in the next step will be $\sigma \otimes c$.
%%
%{\bf A2} checks if $c$ is entailed by  $\sigma$: if not, the computation is blocked.
%%
%Axiom {\bf A3} replaces a procedure identifier with the associated body, renaming the formal parameter with the actual one.
%%$A[^y/_x]$ stands for the agent obtained by replacing all the occurrences of $x$ with $y$.
%%
%%Axiom {\bf A4} hides the variable $x$ occurring in $A$, replacing it  
%%with a globally fresh variable,
%%as ensured by $w \not \in \Delta$.
%%The latter is more general than just requiring that 
%%$w \not \in fv(\exists_x A) \cup sv(\sigma)$, since
%%$\langle B, \rho \rangle   \rarrow_\Delta$ implies that 
%%$fv(B) \cup sv(\rho) \subseteq \Delta$.\footnote{Our rule is  reminiscent of 
%%$(8)$ in~\cite[p.~342]{popl91}.}

In Table~\ref{fig:ALTS} and Table~\ref{fig:CRLTS} we refine the notion of transition (respectively given in Table~\ref{fig:operational} and Table~\ref{fig:operational2})
by adding a label that carries additional information about the constraints that cause the reduction.
Indeed, rules in Table~\ref{fig:ALTS} and Table~\ref{fig:CRLTS} mimick those in Table~\ref{fig:operational} and Table~\ref{fig:operational2}, except for a constraint $\alpha$ that
represents the additional information that must be combined with $\sigma$ in order to fire an action
from $\langle A, \sigma\rangle$  to $\langle A', \sigma' \rangle$.
% i.e., $\langle A, \sigma \otimes \alpha\rangle \longrightarrow \langle A' , \sigma' \rangle$.

For the rules in Table~\ref{fig:ALTS}, as well as for rule {\bf  LR1}, we can restate the intuition given for their unlabelled counterparts. 
The difference concerns the axioms for the $\hbox{\ask}(c)$: if $c$ is not entailed from $\sigma$, then
some additional information  is imported from the environment, ensuring that the state
$\alpha \otimes \sigma \leq c$ allows the execution of $\hbox{\ask}(c)$.

Once again, the more complex axiom is {\bf LR2}. With respect to {\bf R2}, the additional intuition is that 
$\alpha$ should not contain the restricted variable $x$: additional information can be obtained from the environment,
as long as it does not interact with data that are private to the local agent.
%
Note that by choosing $\rho = \sigma_1 \odiv (\alpha \otimes \pi_\otimes  \otimes \sigma_0)$ we are 
removing $\alpha$ from the update to be memorised in the local store. However, 
since $\alpha$ is added to the global store, it will not be necessary to receive it again in the future. 

\begin{remark}
An alternative solution for rule ${\bf LA2}$ would have been to restrict the possible reductions to the one with maximal label, 
that is, $\langle \hbox{\ask}(c) \mapsto A, \sigma \rangle \xrightarrow{c \odiv \sigma} \langle A, (c \odiv \sigma) \otimes \sigma \rangle$. 
However, this might have been restrictive in combination with rule ${\bf LR2}$.
Consider  our running example and the configuration 
$\langle \exists^{[x > 1]}_x \hbox{\ask}(y > 2) \mapsto \hbox{\ostop}, \monid \rangle$. The initial configuration in the premise is
$\langle \hbox{\ask}(y > 2) \mapsto \hbox{\ostop}, x > 1 \rangle$ and $(y > 3) \odiv (x > 1) = (x \leq 1) \vee (y > 2)$.
Selecting $\alpha = (x \leq 1) \vee (y > 2)$ is problematic, since $x$ occurs free. Instead, the choice of $\alpha = (y > 2)$,
or any other value such as $y > 3$,$ y > 4$, $\ldots$, fits well with the intuition of some information  from the environment 
triggering the reduction.

Note instead that the choice of removing the requirement $x \not \in sv(\alpha)$ and put $\exists_x \alpha$ as label in the 
conclusion of rule ${\bf LR2}$ would  be too liberal. It would work in our previous example, since 
$\exists_x((x \leq 1) \vee (y > 3)) = y > 3$. However, consider e.g. 
the configuration $\gamma = \langle \exists^{[x > 1]}_x \hbox{\ask}(x > 2) \mapsto \hbox{\ostop}, \monid \rangle$. 
We would have that 
$\langle \hbox{\ask}(x > 2) \mapsto \hbox{\ostop}, x > 1 \rangle \xrightarrow{x \neq 2} \langle \hbox{\ostop}, x > 2 \rangle$,
and then allowing the reduction $\gamma \xrightarrow{\monid} \langle \exists^{[x > 1, x \neq 2]}_x \hbox{\ostop}, \monid \rangle$,
which clashes with the intuition that receiving information should not enable reductions involving (necessarily) 
the restricted variable.
\end{remark}

\begin{lemma}[On labelled monotonicity]
\label{l-mono}
Let $\langle A, \sigma \rangle \xrightarrow{\alpha} \langle B, \rho \rangle$ be a labelled reduction. 
Then 
%\begin{enumerate}
$\rho = (\rho \odiv (\alpha \otimes \sigma)) \otimes \alpha \otimes \sigma$ and 
$fv(\langle B, \rho \rangle) \subseteq fv(\langle A, \sigma \rangle) \cup sv(\alpha)$.
%\end{enumerate}
Moreover, if $\alpha \neq \monid$ then $\rho \odiv (\alpha \otimes \sigma) = \monid$.
\end{lemma}

\begin {proof}
The first part is immediate as for Lemma~\ref{mono}.
%
For the latter part, note that $\alpha \neq 1$ just means that axioms {\bf LA1}
and {\bf LA3} are never applied in the reduction. Let us check that the property 
holds for {\bf LR2}. Inductively, we know that 
$(\sigma_1 \odiv (\alpha \otimes \pi_\otimes \otimes \sigma_0)) = \monid$, and
$(\alpha \otimes \sigma \otimes \exists_x ( \sigma_1 \odiv (\alpha \otimes \pi_\otimes  \otimes \sigma_0))) \odiv (\alpha \otimes \sigma) = \monid$
immediately follows.
\qed
\end{proof}

\begin{remark}
The lemma before tell us that if $\langle A, \sigma \rangle \xrightarrow{\alpha} \langle B, \rho \rangle$ 
is a labelled reduction and $\alpha \neq \monid$, then $\rho = \alpha \otimes \sigma$.
Indeed, since $\alpha \neq \monid$ its derivation must use the axiom  {\bf LA2}.
%
Consider e.g. a labelled reduction 
$ \langle \exists^\pi_x A, \sigma \rangle \xrightarrow{\alpha}
  \langle \exists^{\pi \rho}_x B, \alpha \otimes \sigma \otimes \exists_x \rho \rangle$.
  %Item $1$ of the lemma above tell us that 
  If $\alpha \neq \monid$, then $\rho = \monid$. 
  Indeed, this is the expected behaviour: if an input from the context is needed,
  there is no contribution by the agent to the local store, hence the update is 
  correctly $\monid$.
\end{remark}

\begin{definition}
A configuration is l-reachable if it can be
reached by an initial configuration via a sequence of 
labelled reductions.
\end{definition}

%We denote as $\xRightarrow{\mu}$ the reflexive and transitive closure of $\xrightarrow{\alpha}$,
%with $\mu \in {\mathcal C}^\ast$ a sequence of elements defined in the expected way: 
%$\xRightarrow{\mu}\ =\ \xrightarrow{\mu} \ldots \xrightarrow{\mu_n}$
%with $\mu = [\mu_1, \ldots, \mu_n]$.

\begin{lemma}[On labelled monotonicity, II]
\label{l-mono2}
Let 
$\langle B \parallel \exists_x^\pi C, \sigma \rangle$ 
be an l-reachable configuration. 
Then 
%\begin{enumerate}
%\item $\exists \sigma'.\  \sigma = \sigma' \otimes \exists_x \pi$
%\end{enumerate}
$\sigma = (\sigma \odiv (\exists_x \pi)_\otimes) \otimes (\exists_x \pi)_\otimes$.
\end{lemma}

\begin{proof}
As for unabelled reductions, it is easy to show that the equivalent property
on the existence of $\sigma'$ such that 
$\sigma = \sigma' \otimes (\exists_x \pi)_\otimes$ is preserved by 
labelled reductions. 
\qed
\end{proof}


\section{On the correspondence between reduction semantics}
\label{corres}
This section shows the relationship between the labelled and the unlabelled reduction semantics.
%We say that a configuration $\gamma$ is reachable if there exists an initial agent $A$ such that
%$\langle A, \sigma \rangle \to^\ast \gamma$, and it is l-reachable if 
%$\langle A, \sigma \rangle \xRightarrow{\mu} \gamma$.
%
In the following, we assume that $\C$ in invertible.

\begin{theorem}[Soundness]
\label{sound}
Let $\langle A, \sigma \rangle$ be an l-reachable configuration and
$\langle A, \sigma \rangle \xrightarrow{\alpha} \langle B, \sigma' \rangle$. 
Then $\langle A, \alpha \otimes \sigma \rangle$ is a reachable configuration and
$\langle A, \alpha \otimes \sigma \rangle \to \langle B, \sigma' \rangle$.
\end{theorem}

\begin{proof}
The property holds for the axioms, since e.g. for {\bf LA2} we know that
$\alpha \leq c \odiv \sigma$ implies $\alpha \otimes \sigma \leq c$ for residuated POMs.
%
We then proceed by induction on rule derivations,
presenting only the proof for rule {\bf LR2}.
%
We have 
$$\langle \exists^\pi_x A, \sigma \rangle \xrightarrow{\alpha}
  \langle \exists^{\pi \rho}_x B, \alpha \otimes \sigma \otimes \exists_x \rho \rangle$$
  with $\sigma_0 = \sigma \odiv (\exists_x \pi)_\otimes$, $\rho = \sigma_1 \odiv (\alpha \otimes \pi_\otimes  \otimes \sigma_0)$, and 
  $x \not \in sv(\sigma) \cup sv(\alpha)$.
The premise tells us that $\langle A, \pi_\otimes \otimes \sigma_0 \rangle$ is l-reachable and
$$\langle A, \pi_\otimes \otimes \sigma_0 \rangle \xrightarrow{\alpha}
  \langle B, \sigma_1 \rangle$$
%with  $\sigma_0 = \sigma \odiv \bigotimes \exists_x \pi$.
%
By induction hypothesis $\langle A, \alpha \otimes \bigotimes \pi \otimes \sigma_0 \rangle$ is reachable and
$$\langle A, \alpha \otimes \pi_\otimes \otimes \sigma_0 \rangle \to
  \langle B, \sigma_1 \rangle$$
Now recall that we must prove that $\langle \exists^\pi_x A, \alpha \otimes \sigma \rangle$ is reachable and
$$\langle \exists^\pi_x A, \alpha \otimes \sigma \rangle \to
  \langle \exists^{\pi \rho}_x B, \alpha \otimes \sigma \otimes \exists_x \rho \rangle.$$
Reachability is implied by $\langle A, \pi_\otimes \otimes \widehat{\sigma_0} \rangle$ being reachable,
%and
%$$\langle A, \bigotimes \pi \otimes \widehat{\sigma_0} \rangle \to
%  \langle B, \widehat{\sigma_1} \rangle$$
 with $\widehat{\sigma_0} = (\alpha \otimes \sigma) \odiv (\exists_x \pi)_\otimes$.
% and for suitable $\widehat{\sigma_1}$.
% as long as $\alpha \otimes \sigma \otimes \exists_x \rho = \alpha \otimes \sigma \otimes \exists_x \widehat{\rho}$
%with $\widehat{\rho} = \widehat{\sigma_1} \odiv (\alpha \otimes \bigotimes \pi  \otimes \widehat{\sigma_0})$.
%
Now we have that 
$$\pi_\otimes \otimes \widehat{\sigma_0} = \pi_\otimes \otimes ((\alpha \otimes \sigma) \odiv (\exists_x \pi)_\otimes)$$ 
$$=^1 \pi_\otimes \otimes ((\alpha \otimes (\sigma \odiv (\exists_x \pi)_\otimes) \otimes (\exists_x \pi )_\otimes) \odiv (\exists_x \pi)_\otimes)$$
$$=^2  (\pi_\otimes \odiv (\exists_x \pi)_\otimes) \otimes (\exists_x \pi)_\otimes \otimes 
((\alpha \otimes (\sigma \odiv (\exists_x \pi)_\otimes) \otimes  (\exists_x \pi)_\otimes) \odiv (\exists_x \pi)_\otimes)$$
$$ =^3 (\pi_\otimes \odiv (\exists_x \pi)_\otimes) \otimes \alpha \otimes (\sigma \odiv (\exists_x \pi)_\otimes) \otimes (\exists_x \pi)_\otimes$$
$$= \alpha \otimes \pi_\otimes \otimes  (\sigma \odiv (\exists_x \pi))_\otimes 
= \alpha \otimes \pi_\otimes \otimes  \sigma_0
$$
and by induction hypothesis reachability follows, as well as 
$$\langle A, \pi_\otimes \otimes \widehat{\sigma_0} \rangle \to \langle B, \sigma_1 \rangle.$$
and we are done. For proving $1$ we used that $\langle \exists^\pi_x A, \sigma \rangle$ is a reachable configuration
and Lemma~\ref{l-mono2}, for $2$ we used that $\C$ is invertible and $\pi_\otimes \leq (\exists_x \pi)_\otimes$,
for $3$ we used the equality $a \otimes ((a \otimes b) \odiv a) = a \otimes b$.
\qed
\end{proof}

The theorem above can be easily reversed, saying that if a configuration $\langle A, \sigma \rangle$ is reachable,
then it is also l-reachable via a sequence of reductions labelled by $\monid$.

\begin{lemma}
Let $\langle A, \sigma \rangle$ be a reachable configuration and
$\langle A, \sigma \rangle \to \langle B, \sigma' \rangle$. 
Then $\langle A, \sigma \rangle$ is an l-reachable configuration and
$\langle A, \sigma \rangle \xrightarrow{\monid}  \langle B, \sigma' \rangle$.
\end{lemma}
\begin{proof}
Only axiom  {\bf LA2} needs to be checked, and it is obvious, since $\sigma \leq c$ implies that 
$\monid \leq c \odiv \sigma$ and thus 
$\langle \hbox{\ask}(c) \mapsto A, \sigma \rangle \xrightarrow{\monid}
  	\langle A, \monid \otimes \sigma \rangle$.
\end{proof}	
	
[RIVEDERE DA QUA]

However, we are interested in a more general notion of completeness, possibly taking into account 
reductions needing a label, as stated in the theorem below.

\begin{lemma}
\label{minor}
Let $\langle A, \tau \rangle$ be a reachable configuration such that
$\sigma \leq \tau$. Then $\langle A, \sigma \rangle$ is a reachable configuration.

Moreover, if  $\langle A, \tau \rangle \to \langle B, \tau' \rangle$ then
$\langle A, \sigma \rangle \to \langle B, \sigma' \rangle$, 
with $\sigma' = \tau' \otimes (\sigma \odiv \tau)$.
\end{lemma}
\begin{proof}
%As for the first item, i
As for the first item, it suffices to note that if
$\langle B, \mu \rangle \to \langle A, \tau \rangle$,
then $\langle A, \sigma \rangle$ is reachable from
 $\langle \hbox{\tell}(\sigma \odiv \tau) \parallel B, \mu \rangle$.
 [DA FINIRE. MI PARE CHE SERVA LA LOCALIZATION]
\end{proof}

\begin{lemma}
\label{riminor}
Let $\langle A, \tau \rangle \xrightarrow{\beta} \langle B, \tau' \rangle$ with $\alpha \leq \beta$. 
%
If $\beta \neq \monid$ then $\langle A, \tau \rangle \xrightarrow{\alpha} \langle B, \tau' \otimes (\alpha \odiv \beta) \rangle$.
%\\
%
%Then $\langle A, \tau \rangle \xrightarrow{\alpha} \langle B', \tau'' \rangle$
%and $\tau'' \leq \tau'$.
\end{lemma}

The lemma just states that the label in a reduction can always be strengthened,
as long as rule  {\bf LA1} is not used in the reduction labelled by $\beta$.
The proof exploits the premise of rule {\bf LA2} together with Lemma~\ref{l-mono},
and the condition $\beta \neq \monid$ is required in the inductive step for rule  {\bf LR2}.

\begin{theorem}[Completeness]
Let $\langle A, \tau \rangle$ be a reachable configuration such that
$\sigma \leq \tau$ and
$\langle A, \sigma \rangle \to \langle B, \sigma' \rangle$. 
Then 
%$\langle A, \tau \rangle$ is a l-reachable configuration and
$\langle A, \tau \rangle \xrightarrow{\alpha} \langle B, \tau' \rangle$
for some $\alpha$ such that $\sigma \odiv \tau \leq \alpha$
and $\sigma' = \tau' \otimes (\sigma \odiv (\tau \otimes \alpha))$.
\end{theorem}

Note that by invertibility $\sigma \odiv \tau \leq \alpha$ implies $\sigma \leq \alpha \otimes \tau$,
thus again by invertibility $\sigma = \tau \otimes \alpha \otimes (\sigma \odiv (\tau \otimes \alpha))$.

\begin{proof}
[DA RIVEDERE]
We proceed by induction on derivation.
The result is immediate for axiom ${\bf A1}$ and ${\bf A3}$, with 
$\alpha = \monid$, and it holds by induction hypothesis for 
${\bf LR1}$.

As for ${\bf A2}$, note that $\sigma' = \sigma \leq c$ and that 
$\langle \hbox{\ask}(c) \mapsto A, \tau \rangle \xrightarrow{\alpha} \langle A, \alpha \otimes \tau \rangle$ 
for any $\alpha \leq c\odiv \tau$, so that $\tau' = \alpha \otimes \tau$.
%
By choosing $\alpha = c \odiv \tau$ we have that $\sigma \leq c$
implies $\sigma \odiv \tau \leq c \odiv \tau = \alpha$,
and 
by invertibility 
$\sigma = \alpha \otimes \tau \otimes  (\sigma \odiv (\tau \otimes \alpha))$.

Let us  move to ${\bf R2}$ and consider the configurations 
$\langle \exists^\pi_x A, \sigma \rangle$  and
$\langle \exists^\pi_x A, \tau \rangle$
such that $x \not \in sv(\sigma) \cup sv(\tau)$. 
Since they are both reachable, the former by Lemma~\ref{minor}, we have that 
$\sigma = (\sigma \odiv (\exists_x \pi)_\otimes) \otimes (\exists_x \pi)_\otimes$ and
$\tau = (\tau \odiv (\exists_x \pi)_\otimes) \otimes (\exists_x \pi)_\otimes$
by Lemma~\ref{mono2}.
%
We have 
$$\langle \exists^\pi_x A, \sigma \rangle \to
  \langle \exists^{\pi \rho}_x B, \sigma \otimes \exists_x \rho \rangle$$
  with $\sigma_0 = \sigma \odiv (\exists_x \pi)_\otimes$ and $\rho = \sigma_1 \odiv (\pi_\otimes  \otimes \sigma_0)$.
The premise tells us that $\langle A, \pi_\otimes \otimes \sigma_0 \rangle$ is reachable and
$$\langle A, \pi_\otimes \otimes \sigma_0 \rangle \to
  \langle B, \sigma_1 \rangle$$
%
% such that $\sigma_1 =  (\sigma_1 \odiv ( \bigotimes \pi \otimes \sigma_0)) \otimes  \bigotimes \pi \otimes \sigma_0$
% by Lemma~\ref{mono}.
Now let us consider $\tau_0 = \tau \odiv (\exists_x \pi)_\otimes$: clearly $\sigma_0 \leq \tau_0$,
so by induction hypothesis
$$\langle A, \pi_\otimes \otimes \tau_0 \rangle  \xrightarrow{\beta}
  \langle B, \tau_1 \rangle$$  for $\beta$ such that 
  $(\pi_\otimes \otimes \sigma_0) \odiv (\pi_\otimes \otimes \tau_0) \leq \beta$
  and $\sigma_1 = \tau_1 \otimes ((\pi_\otimes \otimes \sigma_0) \odiv (\beta \otimes \pi_\otimes \otimes \tau_0))$,
  recalling that moreover
  $\pi_\otimes \otimes \sigma_0 = \pi_\otimes \otimes \tau_0 \otimes ((\pi_\otimes \otimes \sigma_0) \odiv (\pi_\otimes \otimes \tau_0))$.


%Moreover, 
%$\tau_1 = (\tau_0 \odiv (\beta \otimes \bigotimes \pi \otimes \tau_0 )) \otimes  \beta \otimes \bigotimes \pi \otimes \tau_0$
%by Lemma~\ref{l-mono}.
%
%\noindent
Now, $(\bigotimes \pi \otimes \sigma_0) \odiv (\bigotimes \pi \otimes \tau_0) \leq \beta$ implies 
$((\bigotimes \pi \odiv \exists_x \pi) \otimes \sigma) \odiv (\bigotimes \pi \odiv \exists_x \pi) \otimes \tau) \leq \beta$
by invertibility, and then it follows that $\sigma \odiv \tau \leq \beta$
since $a \odiv b \leq (a \otimes c) \odiv (b \otimes c)$ in all residuated POMs.

Since  $x \not \in sv(\sigma) \cup sv(\tau)$ we can assume that $x \not \in sv(\beta)$. Then, we have to prove that 
$$\sigma \otimes \exists_x \rho = \beta \otimes \tau \otimes \exists_x \xi \otimes ((\sigma \otimes \exists_x \rho) \odiv (\beta \otimes \tau \otimes \exists_x \xi))$$
with $\xi = \tau_1 \odiv (\beta \otimes \pi_\otimes \otimes \tau_0)$, as well as $\rho = \xi$.
%
\comment{
\noindent
Now, if $\beta = \monid$ by Theorem~\ref{sound}
and Lemma~\ref{riminor}
we have $\sigma_1 =  \tau_1 \otimes 
((\bigotimes \pi \otimes \sigma_0) \odiv (\bigotimes \pi \otimes \tau_0))$
and by induction hypothesis
$$\langle \exists^\pi_x A, \tau \rangle \xrightarrow{\monid}
  \langle \exists^{\pi \mu}_x B, \tau \otimes \exists_x \mu \rangle$$
with $\mu = \tau_1 \odiv (\bigotimes \pi \otimes \tau_0)$, and we have to prove that
$\sigma = \tau \otimes \sigma''$ and $\sigma \otimes \exists_x \rho = \tau \otimes  \exists_x \mu \otimes \sigma''$
for some $\sigma''$.

Let us choose $\alpha = \sigma \odiv \tau$, noting that 
$x \not \in sv(\alpha)$ by Remark~\ref{remdiv}
and that $\alpha \odiv \beta$ is a witness for $\sigma_2$, since
$$\bigotimes \pi \otimes \tau_0 \otimes \beta \otimes (\alpha \odiv \beta) = 
(\bigotimes \pi \odiv \exists_x \pi)  \otimes  \exists_x \pi \otimes \tau_0 \otimes \alpha =
(\bigotimes \pi \odiv \exists_x \pi) \otimes \sigma$$
$$\sigma_1 = \tau_1 \otimes (\alpha \odiv \beta) ?????$$
%
By Lemma~\ref{riminor} we have that
$$\langle A, \bigotimes \pi \otimes \tau_0 \rangle  \xrightarrow{\alpha}
  \langle B, \tau_1 \otimes (\alpha \odiv \beta) \rangle$$
which implies
  $$\langle \exists^\pi_x A, \tau \rangle  \xrightarrow{\alpha}
  \langle \exists^{\pi \mu}_x B, \alpha \otimes \tau \otimes \exists_x \mu \rangle$$
with $\mu = (\tau_1 \otimes (\alpha \odiv \beta)) \odiv (\alpha \otimes \bigotimes \pi  \otimes \tau_0)$.

\noindent
We must prove that $\sigma = \tau \otimes \alpha \otimes \sigma''$ and 
$\sigma \otimes \exists_x \rho = \alpha \otimes \tau \otimes \exists_x \mu \otimes \sigma''$ for some $\sigma''$.
Since $\alpha = \sigma \odiv \tau$, a possible solution would be $\sigma'' = \monid$,
as long as $\rho = \mu$,
%We should then prove that
%$$\sigma_1 \odiv (\bigotimes \pi  \otimes \sigma_0) = (\tau_1 \otimes (\alpha \odiv \beta)) \odiv 
%(\alpha \otimes \bigotimes \pi  \otimes \tau_0)$$
and the latter holds since
$$\rho = \sigma_1 \odiv (\bigotimes \pi  \otimes \sigma_0) = \tau_1 \otimes \sigma_2 \odiv ((\bigotimes \pi  \odiv \exists_x \pi) \otimes \sigma)$$
$$\mu = (\tau_1 \otimes (\alpha \odiv \beta)) \odiv (\alpha \otimes \bigotimes \pi  \otimes \tau_0) = 
(\tau_1 \otimes (\alpha \odiv \beta)) \odiv ((\sigma \odiv \tau) \otimes (\bigotimes \pi  \odiv \exists_x \pi)  \otimes \tau)$$
}
\qed
\end{proof}


\section{Labelled bisimulation}
We now exploit the labelled reductions in order to define suitable notion of bisimilarity without the upward closure condition.
As it occurs with the crisp language~\cite{pippo} and the soft variant with global variables~\cite{festcatuscia}, 
%and differently from most process calculi~\ref{xxx}, 
barbs cannot be removed from the 
definition of bisimilarity because they cannot be inferred by the reductions.

\begin{definition}[Strong bisimilarity]\label{def:strongbis} A strong bisimulation is a symmetric relation $R$ on configurations such that whenever
%$(\gamma_1,\gamma_2) \in R$ with $\gamma_1 = \langle A, \sigma \rangle$
%and $\gamma_2 = \langle B, \rho \rangle$
$( \langle A, \sigma \rangle,\langle B, \rho \rangle) \in R$
\begin{enumerate}
\item if $\langle A, \sigma \rangle \downarrow_c$ then $\langle B, \rho \rangle \downarrow_c$;
\item if $\langle A, \sigma \rangle \xrightarrow{\alpha} \gamma_1$ then there is $\gamma_2$ such that $\langle B, \alpha \otimes \rho \rangle \to \gamma_2$ 
and $(\gamma_1, \gamma_2) \in R$.
\end{enumerate}
We say that $\gamma_1$ and $\gamma_2$ are  strongly bisimilar ($\gamma_1  \sim \gamma_2$) if there exists a strong  bisimulation 
$R$ such that $(\gamma_1 , \gamma_2 ) \in R$. We write $A \sim_{\mathit{s}} B$ if $\langle A, \monid \rangle \sim \langle B, \monid \rangle$.
\end{definition}

Note that $\langle A, \sigma \rangle \sim \langle B, \rho \rangle$ implies
$\sigma = \rho$, as for saturated bisimilarity.
%This fact does not hold if we move to weak relations. 
%
\comment{
\begin{definition}[Weak bisimilarity]\label{def:weakbis} A weak bisimulation is a symmetric relation $R$ on configurations such that whenever
%$(\gamma_1,\gamma_2) \in R$ with $\gamma_1 = \langle A, \sigma \rangle$
%and $\gamma_2 = \langle B, \rho \rangle$
$( \langle A, \sigma \rangle,\langle B, \rho \rangle) \in R$
\begin{enumerate}
\item if $\langle A, \sigma \rangle \downarrow_c$ then $\langle B, \rho \rangle \Downarrow_c$;
\item if $\langle A, \sigma \rangle \xrightarrow{\alpha} \gamma_1$ then there is $\gamma_2$ such that $\langle B, \rho \otimes \alpha \rangle \Rightarrow \gamma_2$ 
and $(\gamma_1, \gamma_2) \in R$;
\end{enumerate}
We say that $\gamma_1$ and $\gamma_2$ are  weakly bisimilar ($\gamma_1  \approx \gamma_2$) if there exists a  
weak  bisimulation $R$ such that $(\gamma_1 , \gamma_2 ) \in R$. 
We write $A \approx B$ if $\langle A, \monid \rangle \approx \langle B, \monid \rangle$.
\end{definition}

\begin{lemma}[Weak bisimilarity, 2]\label{def:weakbis2}
Weak bisimilarity coincides with the relation 
obtained from Definition~\ref{def:strongbis} by replacing $\to$ with $\Rightarrow$ and $\downarrow_c$ with $\Downarrow_c$.
\end{lemma}

\begin{proposition}
Let $\langle A,\sigma \rangle, \langle B,\rho \rangle$ be configurations 
and $c, d \in \mathcal{C}$.
If $\langle A,\sigma \rangle \approx \langle B,\rho \rangle$
and $\langle A,\sigma \otimes d\rangle \downarrow_c$ then 
then $\langle B, \rho \otimes d\rangle \Downarrow_c$.
\end{proposition}
\begin{proof}
Let $\langle A,\sigma \otimes d\rangle \downarrow_c$, that is, \sigma \otimes d \leq c$.
Since \langle A, \sigma \rangle \approx \langle B, \rho \rangle$ and $\langle A,\sigma \rangle \downarrow_\sigma$,
then there exists  $\langle B', \rho'\rangle$ such that 
 $\langle B, \rho \rangle \D \Rightarrow \langle B', \rho' \rangle$ and 
 $\rho' \leq \sigma$.
 [DA FINIRE]
\qed
\end{proof}
}
%
The feasibility of $\sim$ resides on the fact that it is not 
upward closed by definition. Thanks to the correspondence results in Section~\ref{corres}, 
it can be proved so, as well as a congruence.
%
%Recall that a context $C[\cdot]$ is an agent with an placeholder $\cdot$.

\begin{proposition}
Let $\langle A,\sigma \rangle, \langle B,\rho \rangle$ be configurations and $d \in \mathcal{C}$.
If $\langle A,\sigma \rangle \sim \langle B,\rho \rangle$
then $\langle A,\sigma \otimes d\rangle \sim \langle B,\rho \otimes d \rangle$.
\end{proposition}
\begin{proof}
We need to show that the relation 
$R = \{\langle A,\sigma \otimes d \rangle, \langle B,\sigma \otimes d \rangle \mid \langle A,\sigma \rangle \sim \langle B,\sigma \rangle\}$
is a labelled bisimulation. We then assume that 
$\langle A, \sigma \otimes d \rangle  \xrightarrow{\alpha} \langle A', \sigma' \rangle$: 
we need to prove that there exists $B'$ such that
$\langle B,\alpha \otimes  \sigma \otimes d \rangle  \xrightarrow{} \langle B', \sigma' \rangle$ 
  and $\langle A', \sigma' \rangle R \langle B', \sigma' \rangle$.
By soundness we have that 
$\langle A, \alpha \otimes  \sigma \otimes d \rangle  \xrightarrow{}\langle A', \sigma' \rangle$; 
  by completeness that 
  $\langle A,  \sigma \rangle  \xrightarrow{XXX}\langle A', \sigma'' \rangle$,
  where YYYY.
  Since $\langle A,\sigma \rangle \sim \langle B,\sigma \rangle$, we have that 
  there exists $B'$ such that 
  $\langle B, XXX \otimes \sigma \rangle \xrightarrow{} \langle B', \sigma'' \rangle$
  and $\langle A', \sigma'' \rangle \sim \langle B', \sigma'' \rangle$.
\end{proof}

An immediate consequence of the result above is that $\sim$ is a congruence.

\begin{proposition}
Let $A$, $B$ be agents and $C[\cdot]$ a context.
If $A \sim B$
then $C[A] \sim C[B]$.
\end{proposition}

\section{On the equivalence between bisimulation semantics}

\begin{theorem}
$\sim_{\mathit{s}} \subseteq \sim$. Moreover, if $\mathcal{C}$ is invertible, then the equality holds.
\end{theorem}

%\begin{theorem}
%$\approx_{\mathit{s}}  \subseteq \approx$. Moreover, if $\mathcal{C}$ is invertible, then the equality holds.
%\end{theorem}


\section{Concluding Remarks}\label{sec:conclusion}

\bibliographystyle{splncs03}%splncs
\bibliography{main,softccp}

\end{document}
